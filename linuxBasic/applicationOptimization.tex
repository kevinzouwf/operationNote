\chapter{应用程序优化介绍}

%性能调整离工作所执行的地方越近越好: 最好在应用程序里,应用程序能变得极其复杂,研究应用程序内部通常是应用程序开发人员领域。

对于研究系统性能的人员来说,应用程序性能分析包括配置应用程序最佳利用,归纳应用程序使用系统的方式。理解应用程序的作用,怎么操作,怎么执行,对提升吞吐量,降低延时,提高资源利用率有着极大的作用。
应用程序分析方法

\section{线程状态分析}
线程分析的目的是分辨应用程序线程的时间用在了什么地方,这能用来很快解决某些问题。

\section{I/O剖析}

执行I/O的开销包括初始化缓冲区,系统调用,上下文切换,分配内核元数据,检查进程权限和限制,映射地址到设备,执行内核和驱动代码来执行I/O,以及最后 释放元数据和缓冲区。
从效率上来说,每次I/O传输的数据越多,效率越高。

  增加I/O尺寸是应用程序提高吞吐量的常用策略。考虑到每次I/O的固定开销,一次I/O传输128KB要比128次传输1KB高效得多。尤其是磁盘I/O,由于寻道时间,每次I/O开销都很高。
如果应用程序不需要,更大的I/O尺寸也会带来负面效应,一个执行8KB随机读取的数据按128KB I/O的尺寸运行会慢很多。因为128KB的数据传输能力被浪费了。所以选择一个贴近应用程序所需的I/O尺寸
能降低I/O延时。
