\chapter{redis实战}
RemoteDictionary Server 是一个基于Key-value键值对的持久化数据库数据库存储系统, redis和memcached 缓存服务器很像,但是 redis支持的数据存储类型更丰富,包括 String,list ,set和 zset等. redis可以持久化缓存,还会周期行的把更新的数据写入到磁盘以及把修改的操作记录追加到文件里记录下来,比 memcached更有优势的是redis还支持 master-slave主从同步。

redis 支持两种不同的方式对数据化持久化RDB,AOF。RDB在特定时间对数据集做快照存储,其格式为二进制,AOF是把每一个操作追加到文件中,当服务重启后会按顺序对命令重新执行一次重建数据。当两种持久化都开启时优先使用AOF文件恢复原始数据

\section{常见经验}

1.Master写内存快照,save命令调度rdbSave函数,会阻塞主线程的工作,当快照比较大时对性能影响是非常大的,会间断性暂停服务,所以Master最好不要写内存快照。

2.Master AOF持久化,如果不重写AOF文件,这个持久化方式对性能的影响是最小的,但是AOF文件会不断增大,AOF文件过大会影响Master重启的恢复速度。

3.Master调用BGREWRITEAOF重写AOF文件,AOF在重写的时候会占大量的CPU和内存资源,导致服务load过高,出现短暂服务暂停现象。
4. Master最好不要做任何持久化工作,包括内存快照和AOF日志文件,特别是不要启用内存快照做持久化。如果数据比较关键,某个Slave开启AOF备份数据,策略为每秒同步一次。

5. 为了主从复制的速度和连接的稳定性,Slave和Master最好在同一个局域网内。尽量避免在压力较大的主库上增加从库

6. 为了Master的稳定性,主从复制不要用图状结构,用单向链表结构更稳定,即主从关系为:Master<–Slave1<–Slave2<–Slave3…….,这样的结构也方便解决单点故障问题,实现Slave对Master的替换,也即,如果Master挂了,可以立马启用Slave1做Master,其他不变。

下面是我的一个实际项目的情况,大概情况是这样的:一个Master,4个Slave,没有Sharding机制,仅是读写分离,Master负责写入操作和AOF日志备份,AOF文件大概5G,Slave负责读操作,当Master调用BGREWRITEAOF时,Master和Slave负载会突然陡增,Master的写入请求基本上都不响应了,持续了大概5分钟,

上面的情况本来不会也不应该发生的,是因为以前Master的这个机器是Slave,在上面有一个shell定时任务在每天的上午10点调用BGREWRITEAOF重写AOF文件,后来由于Master机器down了,就把备份的这个Slave切成Master了,但是这个定时任务忘记删除了,就导致了上面悲剧情况的发生,原因还是找了几天才找到的。
将no-appendfsync-on-rewrite的配置设为yes可以缓解这个问题,设置为yes表示rewrite期间对新写操作不fsync,暂时存在内存中,等rewrite完成后再写入。最好是不开启Master的AOF备份功能。
Redis主从复制的性能问题,第一次Slave向Master同步的实现是:Slave向Master发出同步请求,Master先dump出rdb文件,然后将rdb文件全量传输给slave,然后Master把缓存的命令转发给Slave,初次同步完成。第二次以及以后的同步实现是:Master将变量的快照直接实时依次发送给各个Slave。不管什么原因导致Slave和Master断开重连都会重复以上过程。Redis的主从复制是建立在内存快照的持久化基础上,只要有Slave就一定会有内存快照发生。虽然Redis宣称主从复制无阻塞,但由于磁盘io的限制,如果Master快照文件比较大,那么dump会耗费比较长的时间,这个过程中Master可能无法响应请求,也就是说服务会中断,对于关键服务,这个后果也是很可怕的。

单点故障问题,由于目前Redis的主从复制还不够成熟,所以存在明显的单点故障问题,这个目前只能自己做方案解决,如:主动复制,Proxy实现Slave对Master的替换等,这个也是Redis作者目前比较优先的任务之一,作者的解决方案思路简单优雅


\section{redis cluster}

  统计生产上比较大的key ./redis-cli --bigkeys,不要对压力大的实例上运行命令,

针对Redis cluter增加节点并重新分slot槽到新节点
\begin{lstlisting}
yum -y install ruby rubygems
~/redis-3.0.7/src/redis-trib.rb add-node 10.10.32.57:7000 10.10.32.27:7001
~/redis-3.0.7/src/redis-trib.rb add-node --slave 10.10.32.57:7001 10.10.32.27:7000
 
./src/redis-trib.rb reshard --timeout 1600 reshard --from redis_id1 redis_id2  --to redis_new--slots 3000 10.10.32.27:7000
 
\end{lstlisting}
