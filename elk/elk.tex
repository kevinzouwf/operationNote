\chapter{elk 简单介绍}
\section{elasticsearch}
elasticsearch其功能主要用于存储与搜索
curl-XGET 'http://localhost:9200/_nodes'

删除日志, curl -XDELETE 127.0.0.1:9200/fudao_requestlog-2017.12.10

列出所有索引并显示状态与索引大小。 curl -XGET 10.9.199.212:9200/_cat/indices 
\section{logstash}
logstash 是一个轻量,开源的服务端数据处理管道,能够收集来自各种来源的数据,实时转换并将其发送到目标位置。其包含三个部分,input,filter,output,下面分三部分一一介绍


\subsection{输入input}
采集各种样式、大小和来源的数据,数据往往以各种各样的形式,或分散或集中地存在于很多系统中。 Logstash 支持各种输入选择 ,可以在同一时间从众多常用来源捕捉事件,\href{https://www.elastic.co/guide/en/logstash/current/input-plugins.html}{支持的输入plugins},里面有各种详细介绍,这里就简单介绍一个用redis做为input的例子

\begin{lstlisting}
	input {
    redis {
        data_type => "pattern_channel"
        key => "logstash-*"
        host => "192.168.0.2"
        port => 6379
        threads => 5
        type => "redis-test"
    }
}

\end{lstlisting}

\subsection{过滤 filter}
数据从源传输到存储库的过程中,Logstash 过滤器能够解析各个事件,识别已命名的字段以构建结构,并将它们转换成通用格式,以便更轻松、更快速地分析和实现商业价值。利用grok从非结构化数据中派生了结构,从IP地址破译出地理坐标,\href{https://www.elastic.co/guide/en/logstash/current/filter-plugins.html}{filter plugins}

针对不同来源,不同类型数据做不同处理及输出,可以在input中增加type 这个字段,然后在filter,output的时候使用type这个字段来判断来源是什么,怎么处理,怎么输出。也可以使用input过来默认增加的字段来处理。

在logstash里字段都需要使用 [fieldname]来进行处理,字段分为top-level(agent, ip, request, ua, response), nested filed(status, bytes, os)。在指定nested filed的时候可以[ua][os]
\begin{lstlisting}
	filter{
     if[source]=~"ftp.log"{
        grok{
         match=>{
                 "message"=>[
                 "\[%{TIMESTAMP_ISO8601:timestamp}\] ALL AUDIT: User \[%{GREEDYDATA:userId}\]\ %{GREEDYDATA:var} \[%{HOSTNAME:ip}\] %{GREEDYDATA:event}.",
                 "\[%{TIMESTAMP_ISO8601:timestamp}\] ALL AUDIT: User \[%{GREEDYDATA:userId}\]\ %{GREEDYDATA:event} \[%{GREEDYDATA:filename}\]."
							]
				}
				
				
				if[event]=~"retrieving file"{
						add_tag=>["Download"]
					}else if["event"]=~"storing file"{
						add_tag=>["Upload"]
					}else if["event"]=~"has logged in"{
						add_tag=>["Login"]			
				}
				add_tag=>["log_ftp"]
			}
			
			}
     }
\end{lstlisting}

在处理数据的时候经常会用到if判断,判断的表达式有, ==, !=, <, >, <=, >= ,正常 =~, !~, in, not in, and, or,nand, xor,   !意思是否定的意思, 多个表达试可以使用()括起来。
\begin{lstlisting}

if EXPRESSION {
  ...
} else if EXPRESSION {
  ...
} else {
  ...
}
	
\end{lstlisting}

使用mutate来移除一个字段
\begin{lstlisting}
filter {
  if [action] == "login" {
    mutate { remove_field => "secret" }
  }
}
\end{lstlisting}

使用in判断,在add tag后, 可以使用 if "aaa" in [tags]来判断tags里是否有aaa

\begin{lstlisting}
filter {
  if [foo] in [foobar] {
    mutate { add_tag => "field in field" }
  }
  if [foo] in "foo" {
    mutate { add_tag => "field in string" }
  }
  if "hello" in [greeting] {
    mutate { add_tag => "string in field" }
  }
  if [foo] in ["hello", "world", "foo"] {
    mutate { add_tag => "field in list" }
  }
  if [missing] in [alsomissing] {
    mutate { add_tag => "shouldnotexist" }
  }
  if !("foo" in ["hello", "world"]) {
    mutate { add_tag => "shouldexist" }
  }
}
\end{lstlisting}

\subsection{输出 output}
尽管 Elasticsearch 是我们的首选输出方向,能够为我们的搜索和分析带来无限可能,但它并非唯一选择,比如exec,file,hadoop等,详见\href{https://www.elastic.co/guide/en/logstash/current/output-plugins.html}{output plugins}


在output,也支持sprintf format,类似像统计一个状态数量便可以用到这个。平进经常用到的类似时间格式了,path => "/var/log/\%{type}.\%\{+yyyy.MM.dd.HH\}" 详细请参见\href{https://www.elastic.co/guide/en/logstash/current/event-dependent-configuration.html}{在配置中使用事件数据和字段}
\begin{lstlisting}
output {
  statsd {
    increment => "apache.%{[response][status]}"
  }
}
\end{lstlisting}

使用exec做为output,运行自定义脚本,需要先安装插件 \textit{ bin/logstash-plugin install logstash-output-exec}

\begin{lstlisting}
	output {
  if [type] == "abuse" {
    exec {
      command => "iptables -A INPUT -s %{clientip} -j DROP"
    }
  }
}
\end{lstlisting}

如果要在logstash使用geoip 可以安装相应插件\textif{bin/logstash-plugin install logstash-filter-geoip}  并下载最新静态IP地址库到本地。https://dev.maxmind.com/geoip/geoip2/geolite2/  下载GeoLite2 City maxmindDB 放到指定目录

\section{kibana}


