
近日公司准备自已做一个运维管理平台,其中的监控部分,打算调用zabbix api接口来进行展示。经过思考之后,计划获取如下内容:
    1、  获得认证密钥
    2、  获取zabbix所有的主机组
    3、  获取单个组下的所有主机
    4、  获取某个主机下的所有监控项
    5、  获取某个监控项的历史数据
    6、  获取某个监控项的最新数据

计划最后展示框架如下内容(这只是值方面,其它的会再加):
主机组1 ----主机名1---监控项1----当前值
                  ---监控项2----当前值
       ----主机名2----监控项1----当前值
                 ----监控项2----当前值
主机组2 ----主机名3---监控项1----当前值
                  ---监控项2----当前值
       ----主机名4----监控项1----当前值
                  ----监控项2----当前值

进入正题
1.     user.login方法获取zabbix server的认证结果官方地址:https://www.zabbix.com/documentation/2.2/manual/api/reference/user/login
python脚本:
[iyunv@yang python]# cat auth.py
#!/usr/bin/env python2.7
#coding=utf-8
import json
import urllib2
# based url and required header
url = "http://1.1.1.1/zabbix/api_jsonrpc.php"
header = {"Content-Type":"application/json"}
# auth user and password
data = json.dumps(
{
   "jsonrpc": "2.0",
   "method": "user.login",
   "params": {
   "user": "Admin",
   "password": "zabbix"
},
"id": 0
})
# create request object
request = urllib2.Request(url,data)
for key in header:
   request.add_header(key,header[key])
# auth and get authid
try:
   result = urllib2.urlopen(request)
except URLError as e:
   print "Auth Failed, Please Check Your Name AndPassword:",e.code
else:
   response = json.loads(result.read())
   result.close()
print"Auth Successful. The Auth ID Is:",response['result']



python脚本运行结果:
1
2
[iyunv@yang python]# python auth.py
Auth Successful. The Auth ID Is: a0b82aae0842c2041386a61945af1180



curl命令:
1
2
3
curl -i -X POST -H 'Content-Type:application/json' -d '{"jsonrpc":
"2.0","method":"user.login","params":{"user":"admin","password":"zabbix"},"auth":
null,"id":0}' http://1.1.1.1/zabbix/api_jsonrpc.php



curl命令运行结果:
1
{"jsonrpc":"2.0","result":"b895ce91ba84fe247e444817c6773cc3","id":0}




2.     hostgroup.get方法获取所有主机组ID把认证密钥放到脚本中,每次获取数据时都需要认证。此处是获取zabbix server上的所有主机组名称与ID号。
官方地址:https://www.zabbix.com/documentation/2.2/manual/api/reference/hostgroup/get
python脚本:
[iyunv@yang python]# catget_hostgroup_list.py
#!/usr/bin/env python2.7
#coding=utf-8
import json
import urllib2
# based url and required header
url = "http://1.1.1.1/zabbix/api_jsonrpc.php"
header = {"Content-Type":"application/json"}
# request json
data = json.dumps(
{
   "jsonrpc":"2.0",
   "method":"hostgroup.get",
   "params":{
       "output":["groupid","name"],
   },
   "auth":"3c0e88885a8cf8af9502b5c850b992bd", # theauth id is what auth script returns, remeber it is string
   "id":1,
})
# create request object
request = urllib2.Request(url,data)
for key in header:
   request.add_header(key,header[key])
# get host list
try:
   result = urllib2.urlopen(request)
except URLError as e:
   if hasattr(e, 'reason'):
       print 'We failed to reach a server.'
       print 'Reason: ', e.reason
   elif hasattr(e, 'code'):
       print 'The server could not fulfill the request.'
       print 'Error code: ', e.code
else:
   response = json.loads(result.read())
   result.close()
   print "Number Of Hosts: ", len(response['result'])
   #print response
   for group in response['result']:
       print "Group ID:",group['groupid'],"\tGroupName:",group['name']



python脚本执行结果:
[iyunv@yang python]# pythonget_hostgroup_list.py
Number Of Hosts:  12
Group ID: 11    Group Name: DB Schedule
Group ID: 14    Group Name: DG-WY-KD-Server
Group ID: 5     Group Name: Discovered hosts
Group ID: 7     Group Name: Hypervisors
Group ID: 2     Group Name: Linux servers
Group ID: 8     Group Name: monitored_linux
Group ID: 9     Group Name: qsmind
Group ID: 12    Group Name: qssec
Group ID: 13    Group Name: switch
Group ID: 1     Group Name: Templates
Group ID: 6     Group Name: Virtual machines
Group ID: 4     Group Name: Zabbix servers



curl命令:
1
curl -i -X POST -H 'Content-Type:application/json' -d '{"jsonrpc": "2.0","method":"hostgroup.get","params":{"output":["groupid","name"]},"auth":"11d2b45415d5de6770ce196879dbfcf1","id": 0}' http://1.1.1.1/zabbix/api_jsonrpc.php



curl执行结果:
1
{"jsonrpc":"2.0","result":[{"groupid":"11","name":"DBSchedule"},{"groupid":"14","name":"DG-WY-KD-Server"},{"groupid":"5","name":"Discoveredhosts"},{"groupid":"7","name":"Hypervisors"},{"groupid":"2","name":"Linuxservers"},{"groupid":"8","name":"monitored_linux"},{"groupid":"9","name":"qsmind"},{"groupid":"12","name":"qssec"},{"groupid":"13","name":"switch"},{"groupid":"1","name":"Templates"},{"groupid":"6","name":"Virtualmachines"},{"groupid":"4","name":"Zabbixservers"}],"id":0}




3.     host.get方法获取单个主机组下所有的主机ID。根据标题2中获取到的主机组id,把主机组id填入到下边脚本中,就可以获得该主机组下所有的主机id。
官方地址:https://www.zabbix.com/documentation/2.2/manual/api/reference/host/get
python脚本:
[iyunv@yang python]# cat get_group_one.py
#!/usr/bin/env python2.7
#coding=utf-8
import json
import urllib2
# based url and required header
url = "http://1.1.1.1/zabbix/api_jsonrpc.php"
header = {"Content-Type":"application/json"}
# request json
data = json.dumps(
{
   "jsonrpc":"2.0",
   "method":"host.get",
   "params":{
       "output":["hostid","name"],
       "groupids":"14",
   },
   "auth":"3c0e88885a8cf8af9502b5c850b992bd", # theauth id is what auth script returns, remeber it is string
   "id":1,
})
# create request object
request = urllib2.Request(url,data)
for key in header:
   request.add_header(key,header[key])
# get host list
try:
   result = urllib2.urlopen(request)
except URLError as e:
   if hasattr(e, 'reason'):
       print 'We failed to reach a server.'
       print 'Reason: ', e.reason
   elif hasattr(e, 'code'):
       print 'The server could not fulfill the request.'
       print 'Error code: ', e.code
else:
   response = json.loads(result.read())
   result.close()
   print "Number Of Hosts: ", len(response['result'])
   for host in response['result']:
       print "Host ID:",host['hostid'],"HostName:",host['name']



python脚本执行结果:
[iyunv@yang python]# pythonget_group_one.py  
Number Of Hosts:  4
Host ID: 10146 Host Name: DG-WY-KD-3F3B-00
Host ID: 10147 Host Name: DG-WY-KD-3F3B-01
Host ID: 10148 Host Name: DG-WY-KD-3F3B-02
Host ID: 10149 Host Name: DG-WY-KD-3F3B-03



curl命令:
1
2
curl -i -X POST -H'Content-Type: application/json' -d '{"jsonrpc":"2.0","method":"host.get","params":{"output":["hostid","name"],"groupids":"14"},"auth":"11d2b45415d5de6770ce196879dbfcf1","id": 0}' 
http://1.1.1.1/zabbix/api_jsonrpc.php



curl命令执行结果:
1
{"jsonrpc":"2.0","result":[{"hostid":"10146","name":"DG-WY-KD-3F3B-00"},{"hostid":"10147","name":"DG-WY-KD-3F3B-01"},{"hostid":"10148","name":"DG-WY-KD-3F3B-02"},{"hostid":"10149","name":"DG-WY-KD-3F3B-03"}],"id":0}




4.     itemsid.get方法获取单个主机下所有的监控项ID根据标题3中获取到的所有主机id与名称,找到你想要获取的主机id,获取它下面的所有items。
官方地址:https://www.zabbix.com/documentation/2.2/manual/api/reference/item
python脚本:
[iyunv@yang python]# cat get_items.py
#!/usr/bin/env python2.7
#coding=utf-8
import json
import urllib2
# based url and required header
url = "http://1.1.1.1/zabbix/api_jsonrpc.php"
header = {"Content-Type":"application/json"}
# request json
data = json.dumps(
{
   "jsonrpc":"2.0",
   "method":"item.get",
   "params":{
       "output":["itemids","key_"],
       "hostids":"10146",
   },
   "auth":"3c0e88885a8cf8af9502b5c850b992bd", # theauth id is what auth script returns, remeber it is string
   "id":1,
})
# create request object
request = urllib2.Request(url,data)
for key in header:
   request.add_header(key,header[key])
# get host list
try:
   result = urllib2.urlopen(request)
except URLError as e:
   if hasattr(e, 'reason'):
       print 'We failed to reach a server.'
       print 'Reason: ', e.reason
   elif hasattr(e, 'code'):
       print 'The server could not fulfill the request.'
       print 'Error code: ', e.code
else:
   response = json.loads(result.read())
   result.close()
   print "Number Of Hosts: ", len(response['result'])
   for host in response['result']:
       print host
       #print "Host ID:",host['hostid'],"HostName:",host['name']



python脚本运行结果:
[iyunv@yang python]# python get_items.py
Number Of Hosts:  54
{u'itemid': u'24986', u'key_':u'agent.hostname'}
{u'itemid': u'24987', u'key_':u'agent.ping'}
{u'itemid': u'24988', u'key_':u'agent.version'}
{u'itemid': u'24989', u'key_':u'kernel.maxfiles'}
{u'itemid': u'24990', u'key_':u'kernel.maxproc'}
{u'itemid': u'25157', u'key_':u'net.if.in[eth0]'}
{u'itemid': u'25158', u'key_':u'net.if.in[eth1]'}
… …



curl命令:
1
2
curl -i -X POST -H 'Content-Type:application/json' -d '{"jsonrpc":"2.0","method":"item.get","params":{"output":"itemids","hostids":"10146","search":{"key_":"net.if.out[eth2]"}},"auth":"11d2b45415d5de6770ce196879dbfcf1","id": 0}' http://1.1.1.1/zabbix/api_jsonrpc.php
#此处加上了单个key的名称



curl命令执行结果:
1
{"jsonrpc":"2.0","result":[{"itemid":"25154"}],"id":0}



5.     history.get方法获取单个监控项的历史数据根据第4项的获取到的所有items id的值,找到想要监控的那项,获取它的历史数据。
官方地址:https://www.zabbix.com/documentation/2.2/manual/api/reference/history/get
python脚本:
[iyunv@yang python]# catget_items_history.py
#!/usr/bin/env python2.7
#coding=utf-8
import json
import urllib2
# based url and required header
url = "http://1.1.1.1/zabbix/api_jsonrpc.php"
header = {"Content-Type":"application/json"}
# request json
data = json.dumps(
{
   "jsonrpc":"2.0",
   "method":"history.get",
   "params":{
       "output":"extend",
       "history":3,
       "itemids":"25159",
       "limit":10
   },
   "auth":"3c0e88885a8cf8af9502b5c850b992bd", # theauth id is what auth script returns, remeber it is string
   "id":1,
})
# create request object
request = urllib2.Request(url,data)
for key in header:
   request.add_header(key,header[key])
# get host list
try:
   result = urllib2.urlopen(request)
except URLError as e:
   if hasattr(e, 'reason'):
       print 'We failed to reach a server.'
       print 'Reason: ', e.reason
   elif hasattr(e, 'code'):
       print 'The server could not fulfill the request.'
       print 'Error code: ', e.code
else:
   response = json.loads(result.read())
   result.close()
   print "Number Of Hosts: ", len(response['result'])
   for host in response['result']:
       print host
       #print "Host ID:",host['hostid'],"HostName:",host['name']



python脚本执行结果:
[iyunv@yang python]# pythonget_items_history.py
Number Of Hosts:  10
{u'itemid': u'25159', u'ns': u'420722133',u'value': u'3008', u'clock': u'1410744079'}
{u'itemid': u'25159', u'ns': u'480606614',u'value': u'5720', u'clock': u'1410744139'}
{u'itemid': u'25159', u'ns': u'40905600',u'value': u'6144', u'clock': u'1410744200'}
{u'itemid': u'25159', u'ns': u'175337062',u'value': u'2960', u'clock': u'1410744259'}
{u'itemid': u'25159', u'ns': u'202705084',u'value': u'3032', u'clock': u'1410744319'}
{u'itemid': u'25159', u'ns': u'263158421',u'value': u'2864', u'clock': u'1410744379'}
{u'itemid': u'25159', u'ns': u'702285081',u'value': u'7600', u'clock': u'1410744439'}
{u'itemid': u'25159', u'ns': u'231191890',u'value': u'3864', u'clock': u'1410744499'}
{u'itemid': u'25159', u'ns': u'468566742',u'value': u'3112', u'clock': u'1410744559'}
{u'itemid': u'25159', u'ns': u'421679098',u'value': u'2952', u'clock': u'1410744619'}



curl命令:
1
curl -i -X POST -H 'Content-Type:application/json' -d '{"jsonrpc":"2.0","method":"history.get","params":{"history":3,"itemids":"25154","output":"extend","limit":10},"auth":"11d2b45415d5de6770ce196879dbfcf1","id": 0}' http://1.1.1.1/zabbix/api_jsonrpc.php



curl命令运行结果:
1
{"jsonrpc":"2.0","result":[{"itemid":"25154","clock":"1410744134","value":"4840","ns":"375754276"},{"itemid":"25154","clock":"1410744314","value":"5408","ns":"839852515"},{"itemid":"25154","clock":"1410744374","value":"7040","ns":"964558609"},{"itemid":"25154","clock":"1410744554","value":"4072","ns":"943177771"},{"itemid":"25154","clock":"1410744614","value":"8696","ns":"995289716"},{"itemid":"25154","clock":"1410744674","value":"6144","ns":"992462863"},{"itemid":"25154","clock":"1410744734","value":"6472","ns":"152634327"},{"itemid":"25154","clock":"1410744794","value":"4312","ns":"479599424"},{"itemid":"25154","clock":"1410744854","value":"4456","ns":"263314898"},{"itemid":"25154","clock":"1410744914","value":"8656","ns":"840460009"}],"id":0}




6.     history.get方法获取单个监控项最后的值只需把上个脚本中或curl中的limit参数改为1就可。


此时监控项的数据已拿到了,接下来的把它传给前台展示就行了。
