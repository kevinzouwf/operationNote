\chapter{keepalive}

LVS是Linux Virtual Server的简称,也就是Linux虚拟服务器, 是一个由章文嵩博士发起的自由软件项目,它的官方站点是www.linuxvirtualserver.org。现在LVS已经是 Linux标准内核的一部分,在Linux2.4内核以前,使用LVS时必须要重新编译内核以支持LVS功能模块,但是从Linux2.4内核以后,已经完全内置了LVS的各个功能模块,无需给内核打任何补丁,可以直接使用LVS提供的各种功能。

(1)LVS是四层负载均衡,也就是说建立在OSI模型的第四层——传输层之上,传输层上有我们熟悉的TCP/UDP,LVS支持TCP/UDP的负载均衡。因为LVS是四层负载均衡,因此它相对于其它高层负载均衡的解决办法,比如DNS域名轮流解析、应用层负载的调度、客户端的调度等,它的效率是非常高的。
   
(2)LVS的转发主要通过修改IP地址(NAT模式,分为源地址修改SNAT和目标地址修改DNAT)、修改目标MAC(DR模式)来实现。
   
①NAT模式:网络地址转换
