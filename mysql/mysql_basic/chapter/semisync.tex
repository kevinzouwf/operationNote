\section{mysql半同步复制}
默认情况下mysql(5.1,5.5)主从同步是异步的,异步复制可以提供最佳的性能,主库把binlog发给从库,这一动作就结束了,并不会验证从库是否接收完毕,但这会带来风险,主从可能数据不一致。

半同步复制是从服务器接收完主服务器发送的binlog日志文件并写入自己的中继日志里然后会给主服务器一个反馈,告诉对方已经接收完毕,这时主库线程才返回给当前session告知操作完成。当出现超时情况时,源主服务器会暂时切换到异步复制,直到从服务器及时收到信息为止。

\begin{lsting}
cd /usr/local/mysql/lib/plugin
ls semisync_*
mysql> install plugin rpl_semi_sync_master soname'semisync_master.so;
mysql> set global rpl_semi_sync_master_enabled=ON
mysql> install plugin rpl_semi_sync_slave soname'semisync_slave.so;
mysql> set global rpl_semi_sync_slave_enabled=ON
\end{lsting}

主库

rpm_semi_sync_master_enabled = ON  开启半同步复制

rpl_semi_sync_master_timeout = 10000 默认为1000毫秒,如果主库在某次事务中的等待时间超过10秒,刚降级为异步复制,如果主库再次探测到slave 从库恢复了,则会自动再次回来半同步模式

rpl_semi_sync_master_wait_no_slave  表示是否允许master每个事务提交后都要等待slave的接收确认信号,默认为ON,如果为OFF,从库追赶上后也不会开启半同步复制模式,需要手工开启

rpl_semi_sync_master_trace_level = 32  开启半同步复制模式的调试级别默认为32.

从库

rpm_semi_sync_slave_enabled = ON  开启半同步复制

rpl_semi_sync_master_trace_level = 32  开启半同步复制模式的调试级别默认为32.


可以使用  show status like '\%semi\%';来查看半同步状态

当从库stop slvae后,会变成异步,

半同步复制只与io线程有关,也就是说slave从库接收完二进制日志后会给一个master主库一个确认,但它并不会管relay-log中继日志是否执行完。

rpl_semi_sync_master_timeout 超时时间不应太长,不然如果从库出了问题,提交一个事务commit后需要等待从库返回确认信息n毫秒后才会超时断开。。
