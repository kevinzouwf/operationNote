too many connections 

如果max_connections 为500-1000,在大多数场合是可以的,一味的增大,仅会治标不治本,反而还会浪费不必要的内存。合理的调整 wait_timeout 默认是28800秒,一般可以设为100秒。
这是因为,客户端连到到server处理完相应的操作后,要等到28800秒以后才释放内存,如果mysql有大量的闲置连接,不仅会白白消耗内存,而且连接不停累加的话最终会造成too many connections这样的错误

为了缓解压力,mysql重启完数据库后,需要手动执行语句,来把热数据加载到InnoDB_Buffer_Pool缓冲池里预热

select count(*) from User;
select count(*) from Password;

5.6 版本可以把Buffer_Pool缓冲池在mysql showdown 时保存到本地磁盘

innodb_buffer_pool_dump_at_shutdown=1
innodb_buffer_pool_dump_now=1 (手工dump)
innodb_buffer_poll_load_at_startup=1
innodb_buffer_pool_load_now=1 (手工恢复)

尽量使用连接查询来代替子查询,连接查询不需要建立临时表,其速度比子查询要快

子查询是,mysql需要先为内层查询语句的查询结果建立一个临时表,然后外层查询语句才能在临时表中查询记录,查询完毕后,还需要撤销这个临时表。

分析binlog日志

mysqlbinlog --no-defaults --base64-output=decode-rows -v -v mysql-bin.000001 |awk '/###/{if($0~/UPDATE|INSERT|DELETE/)count[$2" "$NF]++} END{for (i in count) print i,"\t",count[i]}'|column -t |sort -k3nr |more

三种日志格式 Statement, Row, Mixed

`
