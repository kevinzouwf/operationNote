\chapter{mysql benchmark}
基准测试是唯一方便有效的,可以学习系统在给定的工作负载下会发生什么的方法。
基准测试用于验证系统的一些假设是否符合实际情况,重现系统异常,测试系统运行情况,模拟更高负载,规划未来业务增长。


\section{基准测试的策略}
基准测试主要有两个策略: 一个是针对整个系统的整体测试,另外是单独测试mysql.这两种策略也称为集成式(full-stack)
以及单组件式(single-component)。

通过对整个系统做集成式测试,可以更真实的提示应用的真实表现,发现各部分之间的缓存带来的影响,以及提示系统瓶颈,
系统瓶颈非总是mysql。

但有时候仅需要对mysql做单组件测试,用来比较不同schema,或查询的性能

\subsection{基准测试的指标}



\begin{enumerate}

\item 吞吐量

吞吐量指的是单位时间内的事务处理数,这一直是经典的数据库应用指标,这类基准测试主要针对在线事务处理(OLTP)的吞吐量
常用的测试单位是每秒事务数(TPS),有些采用每分钟事务数(TPM)

\item 响应时间或者延迟

这个指标用于测试任务所需要的整体时间。根据具体的应用,测试的时间单位可能是微秒,毫秒,秒或者分钟。通过使用百分比响应时间
(percentile response time) 来替代最大响应时间。还有使用图表法,(折线图,散点图)

\item 并发性

并发性是一个非常重要又经常被误解和误用的指标,例如web服务器的并发性更准确的度量指标,是在任意时间有多少同时发生的并发请求,

并发性测试不像是一个结果,更像是设置基准测试的一个属性,并发性测试通常不是为了测试应用能达到的并发度,而是为了测试应用在不同并发下的性能
mysql并发可以使用sysbench指定32,64,128个线程的测试,然后记录mysql数据库的Threads_running状态值。

\item 可扩展性

可扩展指的是,给系统增加一倍的工作,在理想情况下就能获得两位的结果,或者说给系统增加一倍的资源,就可以获得两倍的吞吐量,
同时响应时间也必须在可接受的范围内,,但是大多数系统无法达到理想的纯属扩展,随着压力变化,吞吐量都可能越来越差。
\end{enumerate}

\section{基准测试方法}
基准测试需要避免一些常见的错误,使用真实数据的全集,使用真实数据热点区域,使用真实的分布参数,在多用户场景中,与真实用户行为匹配
在系统预热之后才开始测试,测试时间也要长。

我们可以通过下面脚本来获取mysql状态,和系统性能

\listinginputing{../code/gather.sh}

然后通过下面脚本来对数据进行分析

\listinginputing{../code/analyze.sh}

