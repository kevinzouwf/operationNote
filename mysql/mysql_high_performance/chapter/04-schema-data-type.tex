
良好的逻辑设计和物理设计是高性能的基石,应该根据系统将要执行的查询语句来设计schema,这往往需要权衡各种因素。
这里不讨论如何入门数据库设计,有关数据库设计方面的基础知识可以阅读Clare Churcher的 Beginning Database Design

本章以及接下来三章都将讨论逻辑设计,物理设计和查询执行,以及他们之间的相互作用。

选择优化的数据类型

mysql支持的数据类型非常多,选择正确的数据类型对于获得高性能至关重要,不管存储那种类型的数据,下面几个简单的原则都有助于做出更好的选择。
更小的通常更好:一般情况下,使用正确存储数据的最小数据类型。更小的数据类型通过更快,因为他们占用更少的磁盘,内存和CPU缓存,处理时需要的CPU周期也更少
简单就好:简单数据类型操作通常需要更少的CPU周期,比如,整形比字符操作代价更低,因为字符集和校对规则(排序规则)使字符比较比整弄比较更复杂。(应该使用
mysql内建的类型而不是字符串来存储日期和时间,另外一个应该用整形存储IP地址。
尽量避免NULL,通过情况下最好指定列为NOT NULL,除非真的需要存储NULL值,如果查询中包含为NULL的列,对mysql来说更难优化,因为可为NULL的列使得索引,
索引统计都更复杂,可为NULL的列会使用更多的存储空间,在mysql里也需要特殊处理。当可为NULL的列被索引时,每个索引记录需要一个额外的字节,
如果计划在列上建索引,应该避免设计成可为NULL的列。
值得一提的是,InnoDB使用单位的位(bit)存储NULL值,所以对于稀疏数据(多数为NULL,很少有非NULL值)有很好的空间效率,但这一点不适用于MyISAM,

整数类型:可以使用几种整数类型:TINYINT, SMALLINT, MEDIUMINT, INT, BIGINT,分别使用8,16,24,32,64位存储空间,可以存储的值范围从$-2^{N-1}$ 到
$2^{N-1}-1$,其中N是存储空间的位数,使用整数有可选的UNSIGNED 属性,表示不允许负值,这大致可以使正数的上限提高一倍。
有符号和无符号使用相同的存储空间,并具有相同的性能,因此可以根据实际情况选择合适的类型。
mysql可以为整数类型指定宽度,例如INT(11) 对大多数应用这是没有意义的,它不会限制值的合法范围,只是规定了MYSQL的一些交互工具,用来显示字符的个数。
对于存储和计算来说INT(1) 和INT(20)是相同的。

实数类型,实数是带小数部分的数字,然而,它们不只为了存储小数部分,也可以使用DECIMAL存储比BIGINT还大的整数,mysql即支持精确类型,也支持不精确类型

字符串类型

varchar, char类型

varchar用于存储可变长字符串,需要使用1(列长小于等于255字节)或2个额外字节(大于255记录)字符串的长度,由于行是变长的,在UPdate的时候可能执行变得比原来更长,

char是定长的,mysql总是根据定义的字符串长度分配足够的空间,当存储char值时,mysql会删除所有未尾的空格,char不容易产生碎片,对于非常短的列,char比varchar在存储
空间上也更有效率。

mysql>  create table char_test( char_col CHAR(10));
Query OK, 0 rows affected (0.25 sec)

mysql> INSERT INTO char_test(char_col) values
    -> ('string1'),('   string2'),('string3 ');
Query OK, 3 rows affected (0.04 sec)
Records: 3  Duplicates: 0  Warnings: 0

mysql> SELECT CONCAT("'",char_col,"'") FROM char_test;
+--------------------------+
| CONCAT("'",char_col,"'") |
+--------------------------+
| 'string1'                |
| '   string2'             |
| 'string3'                |
+--------------------------+
3 rows in set (0.02 sec)

mysql> create table varchar_test( varchar_col VARCHAR(10));
Query OK, 0 rows affected (0.02 sec)

mysql> INSERT INTO varchar_test(varchar_col) values
    -> ('string1'),('   string2'),('string3 ');
Query OK, 3 rows affected (0.01 sec)
Records: 3  Duplicates: 0  Warnings: 0

mysql> SELECT CONCAT("'",varchar_col,"'") FROM varchar_test;
+-----------------------------+
| CONCAT("'",varchar_col,"'") |
+-----------------------------+
| 'string1'                   |
| '   string2'                |
| 'string3 '                  |
+-----------------------------+
3 rows in set (0.00 sec)

mysql> show table status\G
*************************** 1. row ***************************
           Name: char_test
         Engine: InnoDB
        Version: 10
     Row_format: Dynamic
           Rows: 2
 Avg_row_length: 8192
    Data_length: 16384
Max_data_length: 0
   Index_length: 0
      Data_free: 0
 Auto_increment: NULL
    Create_time: 2017-09-07 10:25:18
    Update_time: 2017-09-07 10:25:18
     Check_time: NULL
      Collation: latin1_swedish_ci
       Checksum: NULL
 Create_options: 
        Comment: 
*************************** 2. row ***************************
           Name: varchar_test
         Engine: InnoDB
        Version: 10
     Row_format: Dynamic
           Rows: 3
 Avg_row_length: 5461
    Data_length: 16384
Max_data_length: 0
   Index_length: 0
      Data_free: 0
 Auto_increment: NULL
    Create_time: 2017-09-07 10:26:06
    Update_time: 2017-09-07 10:26:27
     Check_time: NULL
      Collation: latin1_swedish_ci
       Checksum: NULL
 Create_options: 
        Comment: 
2 rows in set (0.00 sec)



时间与日期

DATETIME 从1001年到9999年,数度为秒,使用8个字节的在存储空间
TIMESTAMP 保存了从19701月1日年夜以来的秒数,和UNIX时间戳相同,只使用4个字节的存储空间,mysql提供 FROM_UNIXTIME()函数把Unix时间戳转换为日期,UNIX_TIMESTAMP()
把日期转为Unix时间戳。

ip地址应该使用元符号整数来存储,因为IP地址实际上是32位元符号整数,用小数点将地址分成四段方便阅读。
