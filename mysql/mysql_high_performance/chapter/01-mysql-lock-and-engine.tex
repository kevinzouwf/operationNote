\documentclass[cyan]{article}

\usepackage[UTF8]{ctex}
\usepackage{fontspec}
    \setmainfont{Times New Roman}
    \setsansfont{Arial}
    \setmonofont{Courier New}
\usepackage[indentfirst]{xeCJK}
    \setCJKmainfont[BoldFont={SimHei},ItalicFont={KaiTi}]{SimSun}
    \setCJKsansfont{KaiTi}

\zhtitile{mysql 高性能}
\begin{document}
\maketitle
\tableofcontents

\chapter{mysql锁与存储引擎}

每个客户端都会在服务器进程中拥有一个线程,这个连接的查询只会在这个单独的线程中执行,该线程只能轮流
在某个CPU核心或者CPU中运行。服务器会负责缓存线程,因此不需要为每个新建的连接创建或者销毁线程,

mysql会解析查询,并创建内部数据结构(解析树),然后对其进行各种优化,包括重写查询,以及选择适合的索引。
对于select 语句,在解析查询之前,服务器会先检查查询缓存,如果能够在其中打到对应的查询,服务器就不必再执行查询解析
优化和执行的整个过程,而是直接返回查询缓存中的结果集。

\section{mysql锁}

无论何时,只要有多个查询需要在同一时刻修改数据,都会产生并发控制的问题,本节主要讨论两个层面的并发控制,服务器层和存储引擎层。
所谓并发控制就是确保在多个事务同时存取数据库中同一数据时不破坏事务的隔离性和统一性以及数据库的统一性。在处理并发读或者并发写时,可以通过
实现一个由两种类型的锁组成的锁系统来解决问题,这两种类型的锁通常被称为共享锁(shared lock) 排他锁(exclusive lock)
也叫读锁(read lock)和写锁(write lock)

读锁是共享的,或者说是互不阻塞的,多个客户在同一时刻可以同时读取同一个资源,而互不干扰,写锁则是排他的,
也就是说一个写锁会阻塞其他的写锁和读锁,这是出于出于安全策略的考虑,只有这样,才能确保在给定时间里,
只有一个用户执行写入,并防止其他用户读取正在写放的同一资源

\section{锁料度}
一种提高共享资源并发性的方式就是让锁定对象更有选择性。尽量只锁定需要修改的部分数据,而不是所有的资源。更理想的方式
是,只对修改的数据片进行精确的锁定。任何时候,在给定的资源上,锁定的数据越少,则系统的并发程度越高,只要相互之间不发生
冲突即可。
问题是加锁需要消耗资源,锁的各种操作,包括获得锁,检查锁,是否已经解除,释放锁等,都会增加系统的开销。

所谓锁策略,就是在锁的开销和数据的安全性之间寻求平衡,mysql提供了多种选择,每种存储引擎都可以实现自己的锁策略和锁粒度

\textbf{表锁(table lock)}

表锁是mysql中最基本的锁策略,并且是开销最小的策略。它会锁定整张表,一个用户对表进行写操作(插入,删除,更新等)
前需要先获得写锁,这会阻塞其他用户对该表的所有读写操作。写锁比读锁具有更高的优先级,所以一个写锁请求可能会被插入到
读锁队列的前面
虽然存储引擎可以管理自己的锁,mysql还是会使用各种有效的表锁来实现不同的目的,比如ALTER TABLE之类的语句使用表锁,而
忽略存储引擎的锁机制

\textbf{行级锁 (row lock)}

行级锁可以最大程度地支持并发处理(同时也带来了最大的锁开销)。众所周知在InnoDB和XtraDB,以及其他一些存储引擎中实现
了等级锁,


\section{事务}
事务就是一组原子性的SQL查询,如果数据引擎能够成功地对数据库应用该组查询,那么就执行查询,如果其中有任何一条因为崩溃或者其他原因
无法执行,那么所有的语句就不会执行。 也就是经常说的ACID  原子性(atomicity),一致性(consistency),隔离性(isolation),
 持久性(durability),

 \textbf{隔离性}

 在sql中定义了四种隔离级别,每一种都规定了一个事务中所做的修改。那些在事务内和事务间是可见的,那些是不可见的。较低级别的
 隔离通常可以执行更高的并发,系统的开销也更低。

 READ UNCOMMITTED(未提交读)

 在未提交读级别,事务中的修改,即使没有提交,对其他事务也都是可见的,事务可以读取未提交的数据,这也被称为脏读。也会有很多问题,
 所以一般实际应用中很少使用。

READ COMMITTED (提交读)

大多数数据库系统的默认隔离级别都是这个级别,(mysql不是),它满足隔离性的定义,这个级别也可以叫做不可重复读(nonrepeatable read)

REPEATABLE READ (可重复读)

可重复读解决了脏读的问题,该级别保证了同一个事务中多次读取同样记录的结果是一致的。但理论上可重复读隔离级别无法解决幻读问题在InnoDB
XtraDB 存储引擎中通过多版本控制(MVCC )解决幻读问题。

SERIALIZABLE (可串行化)
是最高的隔离级别,它通过强制事务串行执行,避免了幻读问题,SERIALIZABLE对每一行数据都加锁,所以可能导致大量超时和锁争用问题。

死锁是指两个或者多个事务在同一资源上相互占用,并请示锁定对方战胜的资源,从而导致恶性循环的现象,
处理死锁的方法有:将持有最少行级别排他锁的事务进行回滚。

mysql默认是开启事务的

mysql> show variables like 'autocommit';

来查看是否开启自动提交。

mysql> set session transaction isolation level read committed;

来设置隔离级别。


\textbf{MVCC}
多版本并发控制是行级锁的一个变种,但是它很多情况下避免了加锁操作。因此开销更低。虽然实现机制有所不同,但大都实现了非阻塞的读操作
写操作也只锁定必要的行。

\section{存储引擎}

mysql将每个数据库(也可以称之为chema)保存为数据目录下的一个子目录,创建表时,mysql会在数据库子目录下创建一个和表同名的.frm文件保存表的定义。

mysql> show table status

查看表的状态,也可以查看information_schema中对应的表显示表的相关信息。其中有一项Enginx便是该表的存储引擎

\subsection{InnoDB存储引擎}

InnoDB是使用最广泛的存储引擎,它被设计用来处理大量的短期事务,短期事务大部分情况是正常提交的,很少会回滚,InnoDB的性能和自动崩溃恢复特性
使得它在非事务型存储的需求中也很流行。

InooDB采用的是MVCC来支持高并发,并且实现了四个标准的隔离级别,默认级别为REPREATABLE READ(可重复读),并且通过间隙锁(next-key locking)
来防止幻读的出现。

\subsection{MyIsam}

myisam提供大理特性,包括全文索引,压缩,空间函数。但它不支持事务,行级锁。

myisam会将表存储在两个文件中,数据文件和索引文件分别以 .MYD和 .MYI为扩展名, myisam表可以包含动态或者静态行。可以存储记录行数,

如果表创建并导放数据以后,不会再进行修改操作,那么这样的表适合采用压缩表,压缩表不能进行修改(先解压缩,修改,再压缩),压缩表可以减少磁盘占用
因此可以减少磁盘IO,


\subsection{转换存储引擎}

\begin{enumerate}
\item alter table

mysql> alter table mytabl engine = InnoDB;

使用alter table是最简单的方法,可以适合于任何一种存储引擎,但是有一个问题,执行时间很长,myql会按行将数据从原表复制到一张新的表中,
在复制期间可能会消耗系统所有的IO,同时原表还会加上读锁,如果转换表的存储引擎,将会推动和原引擎相关的所有特性

\item 导出与导入

使用mysqldump工具将数据导出到文件,然后修改文件中的CREATE TABLE 语句的存储引擎选项,

\item 创建和查询

先创建一个新的存储引擎的表,然后利用insert select语法来导数据

mysql> CREATE TABLE innodb_table LIKE myisam_table;
mysql> ALTER TABLE innodb_table ENGINE=InnoDB;
mysql> INSERT INTO innodb_table SELECT * FROM myisam_table;

数据量不大的话,这样做工作很好。也可以加上 START TRANSATION;  COMMIT;来提交事务,以便及时回滚

mysql> START TRANSATION;
mysql> INSERT INTO innodb_table SELECT * FROM myisam_table WHERE id BETWEEN x AND y;
mysql> COMMIT;

percona Toolkit 提供了一个pt-online-scheme-change工具,可以比较简单方便的执行上述过程,避免手工操作可能导致的失误和烦琐。
\end{enumerate}
\end{document}
