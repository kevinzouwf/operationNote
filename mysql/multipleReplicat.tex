MySQL多源复制支持【多个低版本->MySQL 5.7】的结构,这样就可以让多个实例的schema汇聚在一台实例上,而且无需升级mysql版本并避免未知风险,用户只需要给这些实例安一个MySQL 5.7作为slave就可以了。

当然Slave必须为MySQL 5.7


本次实验将使用MySQL 5.6.x作为多“主”。

〇 测试环境:
OS:CentOS 6.5
master_1: 192.168.1.185(MySQL 5.6.30)
master_2: 192.168.1.186(MySQL 5.6.30)
slave: 192.168.1.1.187(MySQL 5.7.15)


〇 配置:
master_1相关配置:

[mysqld]
server_id    = 185
log-bin      = master_1
log-bin-index    = master_1.index

master_2相关配置:

[mysqld]
server_id    = 186
log-bin      = master_2
log-bin-index    = master_2.index
slave相关配置:

[mysqld]
server_id    = 187
relay-log    = slave
relay-log-index           = slave.index
# 多源复制结构中的slave,官方要求master-info和relay-log-info存放处必须为TABLE.
# 如果为FILE,则在添加多个master时,会失败:ER_SLAVE_NEW_CHANNEL_WRONG_REPOSITORY.
master-info-repository    = TABLE
relay-log-info-repository = TABLE
〇 为master_1 & master_2上建立复制用户:

GRANT REPLICATION SLAVE ON *.* to repl@'192.168.1.187' IDENTIFIED BY 'repl';
FLUSH PRIVILEGES;
〇 测试数据准备:
master_1测试数据:

master_1> FLUSH LOGS;
Query OK, 0 rows affected (0.00 sec)
master_1> SHOW BINARY LOGS; -- 记住当前binlog的name和position
+-----------------+-----------+
| Log_name        | File_size |
+-----------------+-----------+
| master_1.000001 | 166       |
| master_1.000002 | 455       |
| master_1.000003 | 120       |
+-----------------+-----------+
3 rows in set (0.00 sec)
master_1> CREATE DATABASE master_1;
Query OK, 1 row affected (0.03 sec)
master_2测试数据:

master_2> FLUSH LOGS;
Query OK, 0 rows affected (0.00 sec)
master_2> SHOW BINARY LOGS;    -- 记住当前binlog的name和position
+-----------------+-----------+
| Log_name       | File_size |
+-----------------+-----------+
| master_2.000001 | 166       |
| master_2.000002 | 455       |
| master_2.000003 | 120       |
+-----------------+-----------+
3 rows in set (0.00 sec)
master_2> CREATE DATABASE master_2;
Query OK, 1 row affected (0.02 sec)
〇 在slave上执行:

salve> CHANGE MASTER TO 
    -> MASTER_HOST='192.168.1.185',
    -> MASTER_USER='repl',
    -> MASTER_PORT=3306,
    -> MASTER_PASSWORD='repl',
    -> MASTER_LOG_FILE='master_1.000003',
    -> MASTER_LOG_POS=120
    -> FOR CHANNEL 'master_1';
Query OK, 0 rows affected, 2 warnings (0.02 sec)    -- 此处产生的warnings是一些安全建议和警告,本实验无视。
salve> CHANGE MASTER TO
    -> MASTER_HOST='192.168.1.186',
    -> MASTER_USER='repl',
    -> MASTER_PORT=3306,
    -> MASTER_PASSWORD='repl',
    -> MASTER_LOG_FILE='master_2.000003',
    -> MASTER_LOG_POS=120
    -> FOR CHANNEL 'master_2';
Query OK, 0 rows affected, 2 warnings (0.02 sec)
slave> START SLAVE;
Query OK, 0 rows affected (0.01 sec)
salve> SHOW DATABASES;    -- 此时在master_1和master_2上的binlog events已经被正常的apply了
+--------------------+
| Database            |
+--------------------+
| information_schema |
| master_1           |
| master_2           |
| mysql             |
| performance_schema |
| sys                |
+--------------------+
6 rows in set (0.00 sec)
最后通过start slave status即可查到复制状态

slave> SHOW SLAVE STATUS\G
*************************** 1. row ***************************
               Slave_IO_State: Waiting for master to send event
                  Master_Host: 192.168.1.185
                  Master_User: repl
                  Master_Port: 3306
                ……………………………………………………
             Slave_IO_Running: Yes
            Slave_SQL_Running: Yes
                ……………………………………………………
             Master_Server_Id: 185
                  Master_UUID: ee1f8704-58c4-11e6-95b5-000c297f23b7
             Master_Info_File: mysql.slave_master_info
                    SQL_Delay: 0
          SQL_Remaining_Delay: NULL
      Slave_SQL_Running_State: Slave has read all relay log; waiting for more updates
               ……………………………………………………
                 Channel_Name: master_1
           Master_TLS_Version:
*************************** 2. row ***************************
               Slave_IO_State: Waiting for master to send event
                  Master_Host: 192.168.1.186
                  Master_User: repl
                  Master_Port: 3306
                Connect_Retry: 60
               ……………………………………………………
             Slave_IO_Running: Yes
            Slave_SQL_Running: Yes
               ……………………………………………………
             Master_Server_Id: 186
                  Master_UUID: 53774f2d-7e14-11e6-8900-000c298e914c
             Master_Info_File: mysql.slave_master_info
                    SQL_Delay: 0
          SQL_Remaining_Delay: NULL
      Slave_SQL_Running_State: Slave has read all relay log; waiting for more updates
               ……………………………………………………
                 Channel_Name: master_2
           Master_TLS_Version:
2 rows in set (0.00 sec)
 

〇 测试:
master_1上操作:

master_1> CREATE TABLE master_1.test_table(id int);
Query OK, 0 rows affected (0.05 sec)
master_1> INSERT INTO master_1.test_table SELECT 666666;
Query OK, 1 row affected (0.01 sec)
Records: 1 Duplicates: 0 Warnings: 0
master_2上操作:

master_2> CREATE TABLE master_2.test_table(massage varchar(16));
Query OK, 0 rows affected (0.02 sec)
master_2> INSERT INTO master_2.test_table SELECT '嘿嘿嘿';
Query OK, 1 row affected (0.00 sec)
Records: 1 Duplicates: 0 Warnings: 0
master_2> INSERT INTO master_2.test_table SELECT '三阳之炎';
Query OK, 1 row affected (0.00 sec)
Records: 1 Duplicates: 0 Warnings: 0
slave上操作:

salve> SELECT id FROM master_1.test_table;
+--------+
| id     |
+--------+
| 666666 |
+--------+
1 row in set (0.00 sec)
salve> SELECT massage FROM master_2.test_table;
+--------------+
| massage      |
+--------------+
| 嘿嘿嘿        |
| 三阳之炎      |
+--------------+
2 rows in set (0.00 sec)
 

〇 其他相关语法:

START/STOP/RESET ALL/RESET SLAVE FOR CHANNEL 'XXX';
 
SHOW SLAVE STATUS FOR CHANNEL 'XXX';
ps.
与上述传统position方式类似,GTID方式配置起来也类似,开启GTID后,需要注意使用FOR CHANNEL 'xxx'关键字即可,比如:

CHANGE MASTER TO 
    MASTER_HOST='',
    MASTER_USER='repl',
    MASTER_PORT=3306,
    MASTER_PASSWORD='repl',
    MASTER_AUTO_POSITION = 1
    FOR CHANNEL 'master_1';

多台主机的schema名字不可以一样,(比如master_1为db_00 ... db_09共10库,master_2为db_10 ... db_19,master_3为db_20 ... db_29 ……)



参考文档:
MySQL 5.7 Reference Manual 14 SQL Statement Syntax - 14.4.2.1 CHANGE MASTER TO Syntax
MySQL 5.7 Reference Manual 18 Replication - 18.1.4 MySQL Multi-Source Replication
