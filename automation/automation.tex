\chapter{自动化管理工具}
服务器环境中要想保证其稳定运行,必不可少的便是标准化,自动化,设想任何一个运维人员都是上去手动修改主机配信息,一旦出故障,如果此运维人员还在职,且还记得修改过什么配置,还可以恢复回来,但恢复时长也相当长,这对IT管理造成相当大的困难,公司服务器标准化,自动化势在必行。
\section{ansible}
Ansible 基于python研发的自动化运维工具, ansible是无客户端也不需要启服务端工具,十分方便,主要基于openssl所以安全性也比较高,但是因为任务按队列依次执行,所以并没有saltsatck那样并发的快.特别是维护上百台机子后会感觉到明显慢很多。
\begin{itemize}
\item ansible core : ansible 自身核心模块
\item host inventory: 主机库,定义可管控的主机列表
\item connection plugins: 连接插件,一般默认基于 ssh 协议连接
\item modules:core modules ( 自带模块 ) 、 custom modules ( 自定义模块 )
\item playbooks :剧本,按照所设定编排的顺序执行完成安排任务
\end{itemize}
我们可以使用fetch模块来收集配置文件,在play book里不仅可以指定vars变量,还可以指定vars 文件, var files
\begin{lstlisting}
- hosts: myhosts
  vars_files:
    - default_step.yml
\end{lstlisting}

ansible 经常会配置文件,如果配置变更则需要重启服务,此时需要使用notify模块

\section{salt}

\subsection{salt介绍}
与ansible不同的是,salt是一个C/S架构的软件,salt管理端为master,客户端叫minion,通过server端下发指令,客户端受指令的方式进行操作,saltstack基于zeromq消息队列来管理成千上万台主机客户端,传输指令执行相关操作。采用RSA key方式进行身份确认,传输采用AES方式进行加密,这使得它的安全性得到了保证。

在每个minion启动后便会自动生成RSA公密钥,存入于/etc/salt/pki/minion,中,根据minion配置文件中master地址,主动发送公钥给master等待master接收,master接收后便可以批量管理主机。

\subsection{salt中grains与pillar}

Grains 是saltstack组件之一,记录saltstack Minion 的一些静态信息的组件,(CPU, 内存, 磁盘, 网络, 等) 可以通过grains.items查看某台minion的所有Grain信息,minion的grains信息会在minions启动时汇报给master,在实际应用环境中我们需要根据自己的业务需求去算定义grains,在每次修改完grains后需要同步更新grains.  salt '*' saltutil.sync_grains。 了解更多关于grains函数使用命令查看 salt -E 'client*' sys.list_functions grains

自定义grians有三种方法: 
第一种在/etc/salt/master里直接配置,

\begin{lstlisting}
grains:
  roles:
    - webserver
    - memcache
\end{lstlisting}

第二种在另起一个文件/etc/salt/grains在里面定义,

\begin{lstlisting}
roles:
  - webserver
  - memcache
\end{lstlisting}

第三种使用python定义在minion配置文件中配置grains 放到任何环境中_grains目录下

\begin{lstlisting}
[root@linux-node1 /srv/salt/_grains]# cat my_grains.py
#!/usr/bin/env python
#-*- coding: utf-8 -*-

def my_grains():
    # 初始化一个grains字典
    grains = {}
    grains['iaas'] = 'openstack'
    grains['edu'] = 'sandow'
    return grains
[root@linux-node1 /srv/salt/_grains]# cat roles.py
#!/usr/bin/env python
#-*- coding: utf-8 -*-
import os.path
def roles():
    roles_file= "/etc/salt/roles"
    roles_list= []
    if os.path.isfile(roles_file):
        roles_fd = open(roles_file, "r")
        for eachroles in roles_fd:
            roles_list.append(eachroles[:-1])
    return {'roles': roles_list}
if __name__ == "__main__":
    print roles()
\end{lstlisting}

三种方法优先级为从高到低依次为系统自带,grains文件配置,master grains. 
\subsubsection{Pillar}

% TODO pillar还没有好好学习整理,不知道下次又是啥时候整理了
数据管理中心 Pillar
Pillar 也是salt组件之一,叫数据管理中心,或者说是配置管理中心。 会经常配合states在大规模配置管理工作中使用它,pillar在saltstack中主要的作用就是存储和定义配置管理中需要的一些数据,比如软件版本号,用户名,密码,配置等信息,它的定义存储格式跟grains类似,同样增加完pillar配置后需要刷新 salt '*' saltutil.refresh_pillar。 查看pillar salt '*' pillar.items 
 
master 端配置文件中指定了pillar的文件存放位置,
\begin{lstlisting}
pillar_roots:
 base:
 - /srv/pillar
\end{lstlisting}

同状态模块一样,里面需要有top.sls指定入口文件,编写方式也一样。



grains 与 pillar的区别
名称	存储位置	数据类型	数据采集更新方式	应用
Grains	Minion端	静态数据	Minion启动时收集,也可以使用saltutil.sync_grans进行刷新	存储Mnion基本数据,比如用于匹配Minion,自身数据可以用来做资产管理等。
Pillar	Master端	动态数据	在master端定义,指定给对应的minion,可以使用saltutil.refresh_pillar刷新	存储master指定的数据,只有指定的Minion可以看到。用于敏感数据保存

\subsection{远程执行}

% TODO Target没有细化,salt命令没有讲清 
%运程执行本意是讲 salt这个命令怎么用

有时候仅需要使用salt运行简单的命令,可以使用cmd.run模块  \textbf{salt TARGET cmd.run 'w'}
目标端 target 指定主机名,这里可以使用到正则匹配,grains匹配,pillar匹配,主要组,或者直接列出主机名,这里的匹配在top.sls也可以同样适用

目标可以通过正则来匹配minion id. 或者用grains, pillar subnet/ip address, compound matching, node groups 来匹配


管理对象 target
saltstack系统中我们的管理对象叫作target,在master上我们可以采用不同的target 去管理不同的minion
在target options下可以分很多种匹配方式

分发文件
1. salt-cp 批量分发文件
salt-cp 语法格式为
salt-cp '*' [ options ] SOURCE DEST


\subsection{jinja}


%TODO jinja 应该再细化一下
在编写状态文件的时候经常会引用变量,grains, pillar,这时候就需要使用jinja

变量使用Grains: {{ grains[‘fqdn_ip4’] }}

变量使用执行模块: {{ salt['network.hw_addr']('eth0') }}

变量使用Pillar: {{ pillar[‘apache'][‘PORT'] }}

jinja 模版来写keepalived的优写级

\begin{lstlisting}

- ROUTEID: HAPROXY_MASTER
- STATEID: MASTER
- PRIORITYID: 101

- ROUTEID: HAPROXY_BACKUP
- STATEID: BACKUP
- PRIORITYID: 100

\end{lstlisting}

\begin{lstlisting}


  



{{ motdfile }}:
  file.managed:
    - source: salt://motd

\end{lstlisting}

\subsection{状态模块 state}

状态模块描述minion端的状态,按照官网的说明,往往最强大,最有用的工程解决方案都是基于简单的原则,
(Many of the most powerful and useful engineering solutions are founded on simple principles. Salt States strive to do just that: K.I.S.S. (Keep It Stupidly Simple))

salt state的核心便是sls文件(salt state file) sys文件描述了那些系统应该是什么样子。 sys是以yaml为格式序列化存储数据,所以其本质上就是字典,列表,数字,举个例子

\begin{lstlisting}
apache:
  pkg.installed: []
  service.running:
    - enable: True
    - require:
      - pkg: apache
\end{lstlisting}

这个sls状态文件将会确保apache已经安装,并且已经在运行。第一行apache是这个数据集的ID,全局惟一,一个ID下可以有多个模块,但是不能使用多次使用同一个模块。第二三行表示那些状态模块需要运行。基本模式是\textbf{<state_module>.<function>}, pkg.installed 确定当前主机已经安装了指定软件,如果不指定pkgs则默认安装第一行ID名。第三行service.running表示确保软件已经在运行。如果不指定name,默认以ID为软件名。
最后两行require表示service.running需要依赖于ID为apache下的pkg模块运行完后才会执行。 所以上面可以修改为

\begin{lstlisting}
testpkg:
  pkg.installed:
    - pkgs:
      - httpd
  service.running:
    - name: httpd
    - enable: True
    - require:
      - pkg: testpkg
\end{lstlisting}


要想运行状态文件需要在/etc/salt/master中开始file_roots配置
\begin{lstlisting}
file_roots:
  base:
    - /srv/salt
  dev:
    - /srv/salt/dev/services
    - /srv/salt/dev/states
  prod:
    - /srv/salt/prod/services
    - /srv/salt/prod/states
\end{lstlisting}

base, dev, prod表示环境,salt默认会去base环境下去找状态文件,假设把上面内容保存到/srv/salt/apache/init.sls, 要运行单个sls文件可以使用命令\textbf{salt '*' state.sls apache}运行,这里的apache,salt会去base环境下找apache.sls,如果没有,会继续找有没有目录apache,并且下面有init.sls,如果都没有则返回错误。如果要运行/srv/salt/apache/install.sys 最后的状态文件变成  apache.install 既可



当apache目录下有多个sls时,可以使用include apache.xxx 来引到当前文件中

但是如果环境比较多,不同的主机运行不同的状态文件,你又不想一次次敲命令,又乱又容易弄错怎么办,这时候就出现top.sls,在整个salt状态文件里惟一,他定义了针对不同环境下不同主机运行不同的状态文件。默认放到base环境根目录下,也就是/srv/salt下。
\begin{lstlisting}
base:
  'os:Fedora':
    - webserver
    - match: grain
dev:
  'dev-*':
    - vim
  'db*dev*':
    - db
prod:
  '10.10.200.0/24':
    - match: ipcidr
    - deployments.qa.site1
\end{lstlisting}

最后使用命令\textbf{salt '*' state.highstate},一次搞定。

这里仅列出简单两个模块两个方法的用法,salt有非常多的模块可以使用,可以通过命令来获取所有模块以及模块中的方法及用途

\begin{description}
\item{查看state模块} 'Minion' sys.list_state_modules
\item{查看指定states(git)的所有functions} salt 'Minion' sys.list_state_functions git 
\item{查看指定function的用法} salt 'Minion' sys.state_doc git.config
\end{description}



针对管理对象操作
module 是我们日常使用saltstack最多的一个组件,是用于管理对象操作的,这也是saltstack通过push的方式管理的入口,比如管理日常简单的执行命令,查看包安装情况,查看服务运行情况等都是通过 module来实现的 


为什么要加require?就是因为salt本身是并发的去处理任务,service和pkg有可能现时运行,这样有可能达不到预期结果,所以加着require做前后依赖。除require外,还有很多条件

require: 在执行这一步之前需要满足的东西都列在下面,不满足就不执行,可以依赖整个sls, {'require': [{"sls":"foo"},]},require_in 反过来被谁依赖

watch 里面任何一个状态变化变触发,并不是所有的state都支持watch, service state能支持,watch_in被谁监控

unless执行下面内容,如果结果为False才执行该state, {"unless":["rpm -q vim-enhanced","ls /usr/bin/vim"]}, onlyif 结果返回为True才执行该state

 
onfail, 另一个state执行失败后执行这个, onchanges另一个state执行成功并且产生变化执行,prereq 被要求在xxx state之前执行, use 利用另一个state的参数,
listen/listen_in 和watch/watch_in 类似在所有state最后执行, include 组合多个state, extend 对之前内容扩展, 


\subsection{salt实践}

环境准备,使用两台主机做测试,并且在每一台机子都做hosts解析 

\begin{itemize}
\item master 端 主机名master_101 ip 172.16.1.101
\item minion 端 主机名client_102 ip 172.16.1.102
\end{itemize}

\begin{lstlisting}
#安装软件
yum install salt-master salt-minion -y

#master端
systemctl  start salt-master
systemctl  enable salt-master 

# minion端
sed -i 's/#master: salt/master: 172.16.1.101/g' /etc/salt/minion
systemctl  start salt-minion 
systemctl  enable salt-minion 
# 接收 公钥
salt-key -a client_102 -y
 Accepted Keys:
 client_102
 Denied Keys:
 Unaccepted Keys:
 Rejected Keys:
#测试连通性
salt 'client_102' test.ping
client_102:
True
\end{lstlisting}

使用salt-api

