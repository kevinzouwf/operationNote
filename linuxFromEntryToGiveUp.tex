\documentclass[11pt,a4paper]{book}
\usepackage{xltxtra}
\usepackage[utf8]{inputenc}
\usepackage{fontspec} 
\usepackage{float}
\usepackage{listings} 
\usepackage{xeCJK} 
\usepackage[top=0.8in,bottom=0.8in,left=1.2in,right=0.6in]{geometry}
\usepackage[bookmarksnumbered,colorlinks,linkcolor=black,anchorcolor=black,citecolor=black]{hyperref}
\usepackage{epigraph}
\usepackage{amsmath}
\usepackage{amssymb}
\usepackage{underscore} 
\usepackage{indentfirst}
\setlength{\parindent}{2em}
\lstset{ 
	frame=shadowbox,
	numbers=left,
	numberstyle=\tiny,
	language=python,
	stepnumber=1,
	commentstyle=\small,
	extendedchars=false,
	escapeinside=`` ,
	showstringspaces=false,
	columns=fullflexible,
	breaklines=true,                 % automatic line breaking only at whitespace
	breakautoindent=true,%
        breakindent=4em, %
	captionpos=b,                    % sets the caption-position to bottom
	tabsize=4,
%	basicstyle=\ttm,
}
\setmainfont{Times New Roman}
\setsansfont{Helvetica}
\setmonofont{Courier}
\setCJKmainfont[BoldFont={STXihei},ItalicFont={STKaiti}]{STSong}
%\setCJKsansfont[BoldFont=STXihei]{STKaiti}
%\setCJKmonofont{STSong}

\title{linux从入门到放弃}
\author{蔚雷}
\date{November 28, 1028}

\usepackage{filecontents}
\usepackage{biblatex}
%\usepackage[sectionbib]{chapterbib}


\begin{document}

\maketitle

\tableofcontents 

\mainmatter

\chapter*{Preface} 
运维工作四五年,所做之事已无乐趣,了无生趣,总结运维之事,安装系统,部署服务,改参数,调配置,部署代码,
使用工具大致都相仿,什么shell, kickstart, cobbler, ansible, salt, nginx, apache, php, tomcat, keepalived, mysql, git, svn 
jenkins.等之类软件,用什么都不能说精,也不能说不会,昏昏沉沉一年过去,不闻其道,不见其神,只会其术,
可谓一塌糊涂,这样下去无非是浪费时间,不如借此时机,把所学之事整理成成文,

\part{linux 基础知识}

在第一部门中先了解linux基础操作知识,他们包括系统知识,网络知识,
\chapter{自动化安装系统}


Redhat系主要有两种方式安装系统Kickstart和Cobbler。
Kickstart是一种无人值守的安装方式。它的工作原理是在安装过程中记录人工干预填写的各种参数,并生成一个名为ks.cfg的文件。如果在自动安装过程中出现要填写参数的情况,安装程序首先会去查找ks.cfg文件,如果找到合适的参数,就采用所找到的参数;如果没有找到合适的参数,便会弹出对话框让安装者手工填写。所以,如果ks.cfg文件涵盖了安装过程中所有需要填写的参数,那么安装者完全可以只告诉安装程序从何处下载ks.cfg文件,然后就去忙自己的事情。等安装完毕,安装程序会根据ks.cfg中的设置重启/关闭系统,并结束安装。

Cobbler集中和简化了通过网络安装操作系统需要使用到的DHCP、TFTP和DNS服务的配置。Cobbler不仅有一个命令行界面,还提供了一个Web界面,大大降低了使用者的入门水平。Cobbler内置了一个轻量级配置管理系统,但它也支持和其它配置管理系统集成,如Puppet,暂时不支持SaltStack。

在这之前需要了解几个概念,PXE(pre-boot execution environment) 预启动执行环境,通过网络接口启动计算机,不依赖本地存储设备或本地已安装的操作系统,它的工作模式是client/server工作模式,PXE客户端会调用网际协议IP,用户数据报协议(UDP),动态主机设定(DHCP),小型文件传输协议(TFTP)等网络协议。
PXE工作过程

DHCP(Dynamic Host Configuration Protocol,动态主机配置协议)通常被应用在大型的局域网络环境中,主要作用是集中的管理、分配IP地址,使网络环境中的主机动态的获得IP地址、网关地址、DNS服务器地址等信息,并能够提升地址的使用率。端口号67,安装系统时开启,安装后关闭

 TFTP(Trivial File Transfer Protocol,简单文件传输协议)是TCP/IP协议族中的一个用来在客户机与服务器之间进行简单文件传输的协议,提供不复杂、开销不大的文件传输服务。端口号为69。

\ref{Fig:pxe-step}

\begin{enumerate}
\item PXE客户端(需要安装系统的机子) 通过PXE BOOTROM(自启动芯片)会以UDP发送一个广播,向本网络中的DHCP服务器索取IP
\item DHCP服务端收到客户端的请求,验证是否来至合法的PXE clinet的请求,验证通过它将给客户端一个响应其中包含为客户端分配的IP地址,PXELINUX启动程序(TFTP)位置以及配置文件所在位置
\item 客户端收到服务端的回应后会再请求传送启动所需文件:pxelinux.0, pxelinux.cfg/default. vmlinuz, initrd.img
\item 服务端通过TFTP通讯协议从Boot server下载启动安装程序所必需的文件,然后根据该文件中定义的引导顺序,启动linux安装程序的引导内核
\item 客户端通过pxelinux.cfg/default文件成功引导linux安装内核后,安装程序必须确定通过什么安装介质来安装linux,如果通过网络安培(nfs,ftp,http,),便会初始化网络,并定位安装源位置。此时会读取default文件中指定的自动应答文件ks.cfg所在位置,根据该位置请求下载该文件
\item 从服务端下载完ks.cfg文件后,通过该文件找到os server,并按照文件的配置请求下载安装过程需要的软件包。os server 和客户端建立连接后,将开始传输软件包,客户端将开始安装操作系统。安装完成后将重新引导计算机。
\end{enumerate}

这里有个问题,在第2步和第5步初始化2次网络了,这是由于PXE获取的是安装用的内核以及安装程序等,而安装程序要获取的是安装系统所需的二进制包以及配置文件。因此PXE模块和安装程序是相对独立的,PXE的网络配置并不能传递给安装程序,从而进行两次获取IP地址过程,但IP地址在DHCP的租期内是一样的。

\section{kickstart 安装部署步骤}
服务端环境环境:CentOS release 6.7 (Final)  ip 10.0.0.151 ,selinux,防火墙关闭, 首先安装DHCP,TFTP, HTTP服务

\lstinputlisting{./cobble/codes/installDhcp.sh}

好吧让我们来看看效果如果出现下面画面证明配置成功啦

\ref{Fig:kickstartDhcp}

配置支持PXE的启动程序 syslinux
syslinux是一个功能强大的引导加载程序,而且兼容各种介质。SYSLINUX是一个小型的Linux操作系统,它的目的是简化首次安装Linux的时间,并建立修护或其它特殊用途的启动盘。如果没有找到pxelinux.0这个文件,可以安装一下。

\lstinputlisting{./cobble/codes/installSyslinux.sh}

\subsection{创建ks.cfg文件}
kickstart是为了避免安装操作系统的过程中的交互操作,只要定义好一个kickstar自动应答配置文件ks.cfg,并让安装程序知道该配置文件的位置,就可以在安装中读取配置文件来安装系统。

生成kickstaart配置文件的三种方法:

每安装好一台Centos机器,Centos安装程序都会创建一个kickstart配置文件,记录你的真实安装配置。如果你希望实现和某系统类似的安装,可以基于该系统的kickstart配置文件来生成你自己的kickstart配置文件。(生成的文件名字叫anaconda-ks.cfg位于/root/anaconda-ks.cfg)

Centos提供了一个图形化的kickstart配置工具。在任何一个安装好的Linux系统上运行该工具,就可以很容易地创建你自己的kickstart配置文件。kickstart配置工具命令为redhat-config-kickstart(RHEL3)或system-config-kickstart(RHEL4,RHEL5).网上有很多用CentOS桌面版生成ks文件的文章,如果有现成的系统就没什么可说。但没有现成的,也没有必要去用桌面版,命令行也很简单。

阅读kickstart配置文件的手册。用任何一个文本编辑器都可以创建你自己的kickstart配置文件。

ks.cfg 文件组成大致分为3段

命令段:键盘类型,语言,安装方式等系统的配置,有必选项和可选项,如果缺少某项必选项,安装时会中断并提示用户选择此项的选项

* 软件包段

语法基本可以写成

\begin{lstlisting}
%packages
@groupname:指定安装的包组
package_name:指定安装的包
-package_name:指定不安装的包
* 脚本段(可选)

%pre:安装系统前执行的命令或脚本(由于只依赖于启动镜像,支持的命令很少)
%post:安装系统后执行的命令或脚本(基本支持所有命令)
\end{lstlisting}

首先要使用grub-crypt生成一个密码用于root密码,编写ks配置文件放到/var/www/html/ks_config/CentOS-6.7-ks.cfg,

%\lstinputlisting{./codes/cobble/CentOS-6.7-ks.cfg}

优化脚本,也需要放在下面。/var/www/html/ks_config/optimization.sh

%\lstinputlisting{./codes/cobble/optimization.sh}

编辑default配置文件

\begin{lstlisting}
 vim /var/lib/tftpboot/pxelinux.cfg/default
default ks
prompt 0
label ks
 kernel vmlinuz
 append initrd=initrd.img ks=http://10.0.0.151/ks_config/CentOS-6.7-ks.cfg # 告诉安装程序ks.cfg文件在哪里
# append initrd=initrd.img ks=http://10.0.0.151/ks_config/CentOS-6.7-ks.cfg ksdevice=eth0
# 多了一个参数是为了指定网卡(用于多块多卡的时候)
\end{lstlisting}

知识扩展
PXE配置文件default
由于多个客户端可以从一个PXE服务器引导,PXE引导映像使用了一个复杂的配置文件搜索方式来查找针对客户机的配置文件。如果客户机的网卡的MAC地址为8F:3H:AA:6B:CC:5D,对应的IP地址为10.0.0.195,那么客户机首先尝试以MAC地址为文件名匹配的配置文件,如果不存在就以IP地址来查找。根据上述环境针对这台主机要查找的以一个配置文件就是 /tftpboot/pxelinux.cfg/01-8F:3H:AA:6B:CC:5D。如果该文件不存在,就会根据IP地址来查找配置文件了,这个算法更复杂些,PXE映像查找会根据IP地址16进制命名的客户机配置文件。例如:10.0.0.195对应的16进制的形式为C0A801C3。(可以通过syslinux软件包提供的gethostip命令将10进制的IP转换为16进制)
如果C0A801C3文件不存在,就尝试查找C0A801C文件,如果C0A801C也不存在,那么就尝试C0A801文件,依次类推,直到查找C文件,如果C也不存在的话,那么最后尝试default文件。
总体来说,pxelinux搜索的文件的顺序是:

\begin{lstlisting}
/tftpboot/pxelinux.cfg/01-88-99-aa-bb-cc-dd
/tftpboot/pxelinux.cfg/C0A801C3
/tftpboot/pxelinux.cfg/C0A801C
/tftpboot/pxelinux.cfg/C0A801
/tftpboot/pxelinux.cfg/C0A80
/tftpboot/pxelinux.cfg/C0A8
/tftpboot/pxelinux.cfg/C0A
/tftpboot/pxelinux.cfg/C0
/tftpboot/pxelinux.cfg/C
/tftpboot/pxelinux.cfg/default
\end{lstlisting}


\section{Cobbler}

yum install dhcp tftp-server xinetd httpd cobbler cobbler-web pykickstart –y

\begin{figure}[!ht]
    \centering
     \caption{\label{Fig:pxe-step} pxe step}
    \includegraphics[width=0.8\textwidth]{./cobble/images/pxe-step.png}
\end{figure}

\begin{figure}[!ht]
    \centering
     \caption{\label{Fig:kickstart-dhcp} pxe step}
    \includegraphics[width=0.8\textwidth]{./cobble/images/kickstart-dhcp.png}
\end{figure}

kickstart https://access.redhat.com/documentation/en-us/red_hat_enterprise_linux/6/html/installation_guide/s1-kickstart2-options



在第一部门中先了解linux基础操作知识,他们包括系统知识,网络知识,


\chapter{系统基础}

\section{centos7 安装}

CentOS 7 网卡名不以eth0开始原因,是由于systemd 和 udev 引入了一种新的网络设备命名方式–一致网络设备命名(CONSISTENT NETWORK DEVICE NAMING) 。可以根据固件、拓扑、位置信息来设置固定名字,带来的好处是命名自动化,名字完全可预测,在硬件坏了以后更换也不会影响设备的命名,这样可以让硬件的更换无缝化。带来的不利是新的设备名称比传统的名称难以阅读。比如心得名称是enp5s0.

想要改为像centos6一样以eth0开始的名称有两种方法,在出现安装新系统的时候按下tab键,在kernel启动选项中增加 net.ifnames=0 biosdevname=0 

安装好后需要先做一些优化,fir

\begin{figure}[!ht]
    \centering    
    \includegraphics[width=0.8\textwidth]{linuxBasic/images/centos-bios.png}
    \caption{\label{Fig:async} Asynchronous I/O model}
\end{figure}

当然,如果在安装时忘记操作了也可以在启动后操作。在 /etc/sysconfig/grub下相应位置加上这两个参数,然后再改网卡名既可

\begin{lstlisting}
GRUB_CMDLINE_LINUX=”rd.lvm.lv=vg0/swap vconsole.keymap=us crashkernel=auto  vconsole.font=latarcyrheb-sun16 net.ifnames=0 biosdevname=0 rd.lvm.lv=vg0/usr rhgb quiet”

grub2-mkconfig -o /boot/grub2/grub.cfg

\end{lstlisting}

安装完系统后一般需要关闭selinux, NetworkManager, firewalld, 需要安装net-tools, lsof, tcpdump, epel,

\subsection{桥接}
一、网卡桥接设置:
\begin{itemize}
 \item 网卡配置文件:详见 ifcfg-enp8s0
\item 网桥配置文件: 详见 ifcfg-br0
\end{itemize}

二、网卡绑定设置:
\begin{itemize}
\item 网卡配置文件01:ifcfg-enp6s0f0
\item 网卡配置文件02:ifcfg-enp6s0f1
\item 网桥配置文件: ifcfg-bond0
\end{itemize}

3. 在bond0基础上增加多个桥接网卡
详情参见 ifcfg-bond0.300, ifcfg-virbr300;  ifcfg-bond0.3960, ifcfg-virbr3960

\section{磁盘分区与挂载}

\subsection{磁盘简介}
用于存储数据的物理设备便叫磁盘,磁盘接接口的不同可以分为:IDE,  SATA, SCSI, SAS.

\begin{itemize}
\item IDE的英文全称为“Integrated Drive Electronics”,即“电子集成驱动器”,
\item SCSI的英文全称为“Small Computer System Interface”
\item SATA(Serial ATA)又叫串口硬盘,PC机硬盘的主流趋势。
\item SAS(Serial Attached SCSI)即串行连接SCSI,是新一代的SCSI技术,此接口的设计是为了改善存储系统的效能、可用性和扩充性,并且提供与SATA硬盘的兼容性
\end{itemize}

磁盘内部由多个盘片,机械手臂,磁头,主轴马达组成。在读取数据时主轴马达驱动盘片转动,机械手臂可伸展来让磁头(head)读取数据,在盘片上存储数据,所以磁盘的容量便要看盘片的质量。

\begin{description}
	\item[磁道(Track)]:在每个盘片上由不同半经组成的同心圆叫做磁道,
	\item]扇区(Sector)]:每个磁道中被分隔成的最小的俱单位便是扇区每个扇区为512bytes
	\item[柱面(Cylinder)]:由多个盘片相同磁道所组成的圆柱面便是柱面
\end{description}

磁盘读取与写入数据时会按柱面来写入,只有柱面写完(读完)后才会切换磁道。
磁盘容量计算方式:Head * cylinder * Secor * 512bytes

每个磁盘的第一扇区非常重要,因为在该扇区存放着两个重要信息
1. 主引导分区(Master Boot Record, MBR) 有446bytes 主要用于安装引导加载程序
2. 分区表(Partition table) 记录整块磁盘分区状态,有64bytes

\subsection{磁盘分区}
由于分区表仅有64bytes所以只能记录4组记录区,每组记录区记录了该区段的启始与结束的柱面号码。所以每个磁盘只能分四个主(Primary)或扩展分区(Extended)。
而每个磁盘只允许有一个扩展分区,且扩展分区不能被格式化后存放数据,需要在扩展分区之上划分逻辑分区(Logical)。linux设备中的文件名1-4给主或者扩展分区预留,逻辑分区从5开始。

在linux下给磁盘分区的命令有`fdisk` 适合小于2T的磁盘分区, `parted` 擅长于大于2t 的磁盘分区。分区的实质便是修改分区表

\subsubsection{ fdisk 对磁盘分区}
使用命令`fdisk -cu /dev/sdb` 来进行对sdb分区,用命 -l来查看分区分区时的命令有

\begin{itemize}
\item  d   delete a partition 删除一个分区
\item  n   add a new partition       新增一个分区
\item  p   print the partition table 把分区打印出来
\item  q   quit without saving changes 不保存退出
\item  w   write table to disk and exit  保存分区并退出
\end{itemize}

这里需要注意在交互式分区过程中输错之后需要用**Ctrl + u**来撤消。 交互式实在太费事,这对于批量分区来说太费事可以用下面命令来一键搞定
\begin{lstlisting}
echo -e "n\np\n1\n\n+10G\nn\np\n2\n\n+20G\nw" |fdisk /dev/sdb
\end{lstlisting}

上一便仅是创建两个主分区,第一个分区给10G,第二个分区给20G,创建其它分区也类似

\subsubsection{ parted 对磁盘分区}
parted的操作都是实时的,也就是说你执行了一个分区的命令,他就实实在在地分区了,而不是像fdisk那样,需要执行w命令写入所做的修改, 所以进行parted的测试千万注意不能在生产环境中!
>传统的MBR(Master Boot Record)分区方式,有一个局限:无法支持超过2TB的硬盘的分区(或单个分区超过2TB)如果大于2T就要使用用GPT(Globally Unique Identifier Partition Table Format)分区的概念.

非交互式分区方式

\begin{lstlisting}
parted /dev/sdb  mklabel gpt yes
parted /dev/sdb  mkpart primary ext4 0 100  Ignore
parted /dev/sdb  mkpart primary linux-swap 101 8192 Ignore
parted /dev/sdb  mkpart logical ext4 8193 100GB  Ignore
parted /dev/sdb  mkpart logical ext4 101GB 3000GB Ignore
parted /dev/sdb  quit
\end{lstlisting}

\subsection{ 格式化与挂载}
磁盘分区好后,必须先格式化后才能挂载

\begin{lstlisting}
$ mkfs.ext4 /dev/sdb1
$ mkfs.ext4 /dev/sdb2

$ tune2fs -c -1 /dev/sdb1
tune2fs 1.41.12 (17-May-2010)
Setting maximal mount count to -1
#格式化后便可以挂载了
$ mount /dev/sdb1 /mnt
$ mount |grep --color=auto "/dev/sdb1"
/dev/sdb1 on /mnt type ext4 (rw)
\end{lstlisting}

在这里手动挂载后,系统重启后还需要再手动挂载一次,因为这里需要修改文件/etc/fstab 这个文件以达到开机自动挂载

磁盘被手动挂载之后都必须把挂载信息写入/etc/fstab这个文件中,否则下次开机启动时仍然需要重新挂载。 系统开机时会主动读取/etc/fstab这个文件中的内容,根据文件里面的配置挂载磁盘。这样我们只需要将磁盘的挂载信息写入这个文件中我们就不需要每次开机启动之后手动进行挂载了。挂载的限制
\begin{itemize}
\item  根目录是必须挂载的,而且一定要先于其他mount point被挂载。因为mount是所有目录的跟目录,其他木有都是由根目录 /衍生出来的。
\item  挂载点必须是已经存在的目录。
\item  挂载点的指定可以任意,但必须遵守必要的系统目录架构原则
\item  所有挂载点在同一时间只能被挂载一次
\item  所有分区在同一时间只能挂在一次
\item  若进行卸载,必须将工作目录退出挂载点(及其子目录)之外。
\end{itemize}

下面我们看看看/etc/fstab文件,这是我的linux环境中/etc/fstab文件中的内容

\begin{lstlisting}[language=bash]

$ cat /etc/fstab

#
# /etc/fstab
# Created by anaconda on Wed Oct 28 23:23:38 2015
#
# Accessible filesystems, by reference, are maintained under '/dev/disk'
# See man pages fstab(5), findfs(8), mount(8) and/or blkid(8) for more info
#
UUID=faba0886-9c24-430c-8ce5-f7980c283bbd /                       ext4    defaults        1 1
UUID=a2ea9c91-9424-4d8e-b15f-946ef8413877 /boot                   ext4    defaults        1 2
UUID=f11549e2-cd8a-4ec5-92ca-e8a83a16c87e swap                    swap    defaults        0 0
tmpfs                   /dev/shm                tmpfs   defaults        0 0
devpts                  /dev/pts                devpts  gid=5,mode=620  0 0
sysfs                   /sys                    sysfs   defaults        0 0
proc                    /proc                   proc    defaults        0 0
/dev/sdb1               /mnt                    ext4    defaults        0 0

\end{lstlisting}

可以看到fstab里一共有六列。

第一列 Device **Device**  是磁盘设备文件或者该设备的Label或者UUID
Label就是分区的标签,在最初安装系统是填写的挂载点就是标签的名字。可以通过查看一个分区的superblock中的信息找到UUID和Label name。
例如我们要查看/dev/sda1这个设备的uuid和label name
使用设备名称(/dev/sda)来挂载分区时是被固定死的,一旦磁盘的插槽顺序发生了变化,就会出现名称不对应的问题。因为这个名称是会改变的。不过使用label挂载就不用担心插槽顺序方面的问题。不过要随时注意你的Label name。至于UUID,每个分区被格式化以后都会有一个UUID作为唯一的标识号。使用uuid挂载的话就不用担心会发生错乱的问题了。

\begin{lstlisting}[language=bash]
$ dumpe2fs -h /dev/sda1
dumpe2fs 1.35 (28-Feb-2004)
Filesystem volume name:   /boot   #这个就是Label name
Last mounted on:
Filesystem UUID:          3b10fe13-def4-41b6-baae-9b4ef3b3616c    #UUID
Filesystem magic number:  0xEF53
Filesystem revision #:    1 (dynamic)
Filesystem features:      has_journal ext_attr resize_inode dir_index filetype needs_recovery sparse_super
Default mount options:    (none)
Filesystem state:         clean
#简单点的方式我们可以通过下面这个命令来查看
$ blkid /dev/sda1
/dev/sda1: LABEL="/boot" UUID="3b10fe13-def4-41b6-baae-9b4ef3b3616c" SEC_TYPE="ext3" TYPE="ext2"
\end{lstlisting}


第二列:Mount point 设备的挂载点,就是你要挂载到哪个目录下。

第三列: filesystem 磁盘文件系统的格式,包括ext2、ext3、ext3、reiserfs、nfs、vfat等. 生产场景中如果是大量小文件业务 首选 reiserfs。而 ext4 适合视频下载,流媒体,数据库,小文件业务

 ReiserFS是一个基于B状树的文件系统,拥有非常好的总体性能,特别是对于大量小文件。ReiserFS 拥有良好的伸缩性并具有日志功能。但该文件系统不再受到积极开发,不支持SELinux,基本上已被 Reiser4 取代。ReiserFS文件系统多年来一直用作一些发行版(包括SUSE)的默认文件系统,但现在用得少了。

 XFS文件系统拥有日志功能,包含一些健壮的特性,并针对可伸缩性进行了优化。XFS在RAM中强制缓存中转数据,因此如果使用 XFS,建议采用不间断电源供应。淘宝的数据库在使用此文件系统。

第四列:parameters 文件系统的参数

\begin{description}
	\item[Async/sync]设置是否为同步方式运行,默认为async
	\item[auto/noauto]当挂载mount -a 的命令时,此文件系统是否被主动挂载。默认为auto
	\item[rw/ro      ]是否以以只读或者读写模式挂载
	\item[exec/noexec]限制此文件系统内是否能够进行"执行"的操作
	\item[user/nouser]是否允许用户使用mount命令挂载
	\item[suid/nosuid]是否允许SUID的存在
	\item[Usrquota	]启动文件系统支持磁盘配额模式
	\item[Grpquota	]启动文件系统对群组磁盘配额模式的支持
	\item[Defaults	]同事具有rw,suid,dev,exec,auto,nouser,async等默认参数的设置
\end{description}

第五列:能否被dump备份命令作用, dump是一个用来作为备份的命令。通常这个参数的值为0或者1, 0代表不要做dump备份, 1代表要每天进行dump的操作, 2 代表不定日期的进行dump操作

第六列 是否检验扇区 开机的过程中,系统默认会以fsck检验我们系统是否为完整(clean). 0 不要检验,1 最早检验(一般根目录会选择),2 1级别检验完成之后进行检验


 以上会用到的命令会另一篇文章专门介绍下面仅罗列一些相关的命令
\begin{description}
	\item[格式]:mkfs, tune2fs, dumpe2fs
	\item[挂载]: mount umount /etc/fstab
	\item[磁盘检查], df, fsck,  e2fsck
	\item[调整文件大小] resize2fs
	\item[分区]: fdisk parted, partprobe, dd
\end{description}


 给swap增加容量

\begin{lstlisting}[language=bash]
dd if=/dev/zero  of=/tmp/swap bs=1M count=128
mkswap  /tmp/swap
swapon  /tmp/swap
\end{lstlisting}


机械磁盘读写磁盘数据的原理小结:
1. 磁盘是按照柱面为单位读写数据的, 既先读取同一个盘面的某一个磁道,读完之后如果数据没有读完,磁头也不会切换到其他
的磁道, 而是选择切换磁头,读取下一个盘面相同半径的磁道, 直到所有盘面的相同半径的磁道读取完成之后,如果数据还没有读写成,才会切换其他不同半径的磁道,这个切换磁道的过程称为寻道。
2. 不同磁头间的切换是电子切换, 而不同磁道间的切换需要磁头做径向动动,这个径向运动需要不进行电机调节,这个运行是机械的切换。

第一个硬盘   第二个硬盘  第三个硬盘 , 使用硬件RAID, LVM等工成一个或者多个虚拟磁盘,在系统中以块设备名体现/dev/sda

/dev/sdb  等, 在进行使用之前需要进行格式化(创建虚拟文件系统,不同系统使用的文件系统不一样,xfs,ext3,ext4),每一个分区都有各自的inode与block.以供系统使用。

英文单词, Head磁头, Sector扇区, Track 磁道,Cylinder柱面,Units单元块, Block数据块, Inode索引节点
buffer: 一般 用于写操作, 写缓冲

Raid0是条带化,把多个磁盘合起来组成一个大磁盘 支持1 块到多块盘,容量是所有磁盘之和,读写速度最快,没有冗余
Raid1 只支持偶数盘,镜像盘。 读写性能一般,成本高
Raid5 奇偶校验盘,最少三块,可以块一块盘,写入恨不能不高
Raid10,先做Raid1然后再做Raid0,保存备份以及数据量,最少4块盘,读写性能快,成本高。


\section{文件描述符及通配符}

\subsection{ 通配符}
  普通命令都可以用的特殊符号,不同的通配符有不同的意义,现在简单介绍在linux中不同的通配符的不同意义。
%
%符号  |意义   |符号  |意义    |符号  |意义|
%|:----|:------|:----|:------|:----|:-----|
%|*     |所有   |?     | 代表一个字符|; |命令分隔符|
%|#     |注释   |竖线  |  管道 |~ | 用户家目录|
%|-     |上一次目录|  $|  调用变量| /|  路径分隔符|
%|>  >> |重定向,追加重定向|<  <<| 输入重定向,追加输入重定向|{}    |内容序列|
%|'     |无变量转换功能    |"    |里面变量可以转换| `(反引号) | 把里面内容当做命令执行|

简单举例
\begin{lstlisting}
	mkdir /etc/{bbc, blog}
	echo {a..z}

	a=1
	printf "$a\n"     #输出是1
	printf '$a\n'     #输出结果是 $a
	echo `date`       #输出结果是当时时间长格式
\end{lstlisting}

i_link 硬链接
i_count 进程


删除 文件需要看所在目录是否有写权限,没有写权限时是无法删除目录下面的文件

\section{文件权限}
chmod  只有文件的属主或root 才能来改变文件权限。

创建目录默认755
创建文件默认644

umask  修改默认权限通过八进制的数值来定义用户 创建文件或目录的默认权限

sed -n '65.69p' /etc/bashrc

666-umask
若umask部分位为基数,那么在结果的相应位置加一

777-umask

umask 对应数值表示的是禁止的权限,


特殊权限位(用户权限位)

以下内容不重要:

suid   s(x)    S 4
sgid   s(x)    S 2
sticky t(x)    T 1  沾滞位只能用ROOT来删除或创建。被创建的目录,任何用户可以在该目录下可以创建文件目录,但不能查看其它用户的内容,(/tmp)

授权方法 chmod (4000|2000|1000) /bin/rm  chmod (u|g|o)+(s|t )


seuid  权限位
当二进制命令执行
修改的是命令而非文件
仅对二进制命令才有作用。
suid权限仅在程序命令执行过程中有效。
suid是比较危险的功能,

sgid 是针对用户 组权限位的
对文件来说,sgid的功能如下

sgid 仅对二进制的命令程序有效
二进制命令或程序需要有可执行权限
执行命令在任意用户 可以获得该命令程序执行期间所属组的权限。

sgid 针对 目录
创建一个目录,要求在其 目录下创建的文件或目录的group继承该目录组
chmod 2755 /home/admins/


设置seuid
chmod 4755 /bin/rm

find / -perm 4755 -type f



更改文件属主与组 chown  chgrp


chown owrner:group   dirctory|file
chown :group       dirctory|file
chown owrner       dirctory|file

 groupadd test -g 501


\section{shell}
可执行文件开头第一行一般我们会指定用什么解释器来执行该文件比如shell角本的文件开头一般会加\#! /bin/sh


\subsection{shell定义变量以及调用变量}

运行shell 时会遇到三种变量
1. 局部变量, 在脚本或命令中定义,仅在当前shell实例中有效,其他shell启动的程序不能访问局部变量。
2. 环境变量,  所有的程序,包括shell启动的程序,都能访问环境变量,有些程序需要环境变量来保证其正常运行。必要的时候shell脚本也可以定义环境变量。
3. shell变量, 是由shell程序设置的特殊变量。shell变量中有一部分是环境变量,有一部分是局部变量,这些变量保证了shell的正常运行

定义变量时,变量名开始必须以[a-zA-Z]开始,中间不可以有空格或标点符号(可以用“\_”),变量名不可以使用bash的关键字。
调用变量,只需要在变量名前加"\$"便可以了,考虑到解释器识别边界的问题,一般我们会在变量名外加大括号来确定变量名
删除变量可以用 `unset` 来取消变量的定义 .

现在我们便创建一个test.sh文件并且给它执行权限.可以做测试大括号(花括号)是为了让解释器识别变量名的边界,如果不加的话变量名就成了 \$myAgeyears 这个变量名为空,输出来的便只有 Todey, I'm old  这样与期望值并不一样。

\subsection{在shell中的特殊变量名} 

\begin{tabular}{cl}
变量 & 含义 \\
\$0  & 当前脚本的文件名 \\
\$n  & 传递给脚本或函数的参数。n 是一个数字,表示第几个参数,例 \$1 。 如果超过10便需要写成 \$\{10\} \\
\$\#  & 传递给脚本或函数的参数个数。 \\
\$*  & 传递给脚本或函数的所有参数。 \\
\$@  & 传递给脚本或函数的所有参数。被双引号(" ")包含时,与 \$* 稍有不同,下面将会讲到。 \\
\$?  & 上个命令的退出状态,或函数的返回值。 \\
\$\$  & 当前Shell进程ID。对于 Shell 脚本,就是这些脚本所在的进程ID。 \\
\end{tabular}

现在我们接着在test.sh里第17-22行内容,执行\textbf{./test.sh hello world}后会打印的内容为
\begin{lstlisting}
./tesh.sh
hello
world
hello world
hello world
2
\end{lstlisting}
\subsection{变量赋值与转换} 

\begin{tabular}{ll}
形式  & 说明 \\
\${var} & 变量本来的值 \\
\${var:-word} & 如果变量 var 为空或已被删除(unset),那么返回 word,但不改变 var 的值。 \\
\${var:=word} & 如果变量 var 为空或已被删除(unset),那么返回 word,并将 var 的值设置为 word。 \\
\${var:?message} & 如果变量 var 为空或已被删除(unset),那么将消息 message 送到标准错误输出,可以用来检测变量 var 是否可>以被正常赋值。 若此替换出现在Shell脚本中,那么脚本将停止运行。 \\
\${var:+word} & 如果变量 var 被定义,那么返回 word,但不改变 var 的值。
\end{tabular}
\subsection{shell里的运算} 
在原生bash中不支持简单的数学运算,但是可以通过其他命令来实现,例如 awk 和 expr,expr 最常用。
在使用expr时的格式为 `expr 1 + 2 `
关系运算

\begin{tabular}{lll}
运算符 & 说明 & 举例 \\
-eq & 检测两个数是否相等,相等返回 true &  [ \$a -eq \$b ] 返回 true \\
-ne & 检测两个数是否相等,不相等返回 true &  [ \$a -ne \$b ] 返回 true \\
-gt & 检测左边的数是否大于右边的,如果是,则返回 true &  [ \$a -gt \$b ] 返回 false \\
-lt & 检测左边的数是否小于右边的,如果是,则返回 true &  [ \$a -lt \$b ] 返回 true \\
-ge & 检测左边的数是否大等于右边的,如果是,则返回 true &  [ \$a -ge \$b ] 返回 false \\
-le & 检测左边的数是否小于等于右边的,如果是,则返回 true &  [ \$a -le \$b ] 返回 true。
\end{tabular}

逻辑运算

\begin{tabular}{lll}
运算符 & 说明 & 举例 \\
!  & 非运算,表达式为 true 则返回 false &  [ \! false ] 返回 true \\
-o & 或运算,有一个表达式为true则返回true &  [ \$a -lt 20 -o \$b -gt 100 ] 返回 true \\
-a & 与运算,两个表达式都为true才返回true &  [ \$a -lt 20 -a \$b -gt 100 ] 返回 false \\
\end{tabular}

字符串运算符

\begin{tabular}{lll}
运算符 & 说明 & 举例 \\
=   & 检测两个字符串是否相等,相等返回 true &  [ \$a = \$b ] 返回 false。  \\
!=  & 检测两个字符串是否相等,不相等返回 true &  [ \$a != \$b ] 返回 true  \\
-z  & 检测字符串长度是否为0,为0返回 true &   [ -z \$a ] 返回 false   \\
-n &  检测字符串长度是否为0,不为0返回 true &   [ -n \$a ] 返回 true  \\
str &  检测字符串是否为空,不为空返回 true &  [str \$a ] 返回 true   \\
\end{tabular}

文件测试运算符

\begin{tabular}{lll}
操作符 & 说明 & 举例 \\
-b file &  检测文件是否是**块设备**文件,如果是,则返回 true &  [ -b \$file ] 返回 false  \\
-c file & 检测文件是否是**字符设备**文件,如果是,则返回 true &  [ -b \$file ] 返回 false  \\
-d file & 检测文件是否是**目录**,如果是,则返回 true &  [ -d \$file ] 返回 false  \\
-f file & 检测文件是否是**普通文件**(既不是目录,也不是设备文件),如果是,则返回 true &  [ -f \$file ] 返回 true。 \\
-g file & 检测文件是否**设置了SGID位**,如果是,则返回 true &  [ -g \$file ] 返回 false \\
-k file & 检测文件是否设置了**粘着位(Sticky Bit)**,如果是,则返回 true &  [ -k \$file ] 返回 false \\
-p file & 检测文件是否是具名**管道**,如果是,则返回 true &  [ -p \$file ] 返回 false \\
-u file & 检测文件是否设置了**SUID位**,如果是,则返回 true &  [ -u \$file ] 返回 false。\\
-r file & 检测文件是否**可读**,如果是,则返回 true & [ -r \$file ] 返回 true \\
-w file & 检测文件是否**可写**,如果是,则返回 true &  [ -w \$file ] 返回 true \\
-x file & 检测文件是否**可执行**,如果是,则返回 true &  [ -x \$file ] 返回 true \\
-s file & 检测文件是否**为空**(文件大小是否大于0),不为空返回 true &  [ -s \$file ] 返回 true \\
-e file & 检测文件(包括目录)**是否存在**,如果是,则返回 true &  [ -e \$file ] 返回 true
\end{tabular}

\subsection{shell 里处理字符}

先定义一个变量:\textbf{string="sandow is a gentleman"}

我们可以计算字符串长度 \textbf{echo \$\{\#string\}},

或者实现切片 \textbf{\$\{string:1:4\}}幸运的是这里index都是以0开头。
也可以查找字符 \textbf{expr index "\$string" sandow}

\subsection{shell 中的数组}

在shell中只可以建立一维数组,并且index从0开始,创建数组用小括号。类似这样array_name=(value0 value1 value2 value3)

读取数组中某个值时可以用  \${array_name[index]} 读取所有值可以用 “*” 或者 “@” 计算长度仅需要要array_name前加 “\#” 与之前一样

\subsection{shell 中的if 判断语法}

\begin{lstlisting}
if [ expression 1 ]
then
   Statement(s) to be executed if expression 1 is true
elif [ expression 2 ]
then
   Statement(s) to be executed if expression 2 is true
elif [ expression 3 ]
then
   Statement(s) to be executed if expression 3 is true
else
   Statement(s) to be executed if no expression is true
fi
\end{lstlisting}

\subsection{基本方法} 
case 与excel里的case类似,取值先匹配每一个模式,模式匹配后,刚执行匹配模式相应命令,而不会继续其他模式。如果无一匹配模式,使用星号“\*”
来捕获该值,再执行后面的命令。 case的值后面必须为 `关键字 in`,每一模式必须以左括号结束,取值可以为变量或常数。匹配发现聚会符合某一模式后,其间所有命令开始执行直至遇到`;;`,结束。

\begin{lstlisting}
case var in
pattern1)
    command1
    command2
    command3
    ;;
pattern2)
    command1
    command2
    command3
    ;;
*)
    command1
    command2
    command3
    ;;
esac
\end{lstlisting}

for 语法

\begin{lstlisting}
for 变量 in 列表
do
    command1
    command2
    ...
    commandN
done
\end{lstlisting}

while 语法

\begin{lstlisting}
while command
do
   Statement(s) to be executed if command is true
done
\end{lstlisting}

 until 语法

\begin{lstlisting}
until command
do
   Statement(s) to be executed until command is true
done
\end{lstlisting}

函数语法,function 可有可无,不过做为一个合格的编程人员,有必要加上的。这才是规范。

\begin{lstlisting}
function function_name () {
    list of commands
    [ return value ]
}
\end{lstlisting}

向函数内传递文件和上面一样 `function_name p1 p2 p3` 然后在函数内部用  \$n 调用






\chapter{资源迁移}

\section{磁盘网络复制}

dd是一个非常强大的命令,可以直接复制块文件,经常用于磁盘复制,其复制速度比一般copy,mv快的多。曾经使用此命令恢复引导分区。


dd复制整个磁盘  dd if=/dev/sda of=/dev/sdb. 如果要通过网络把本地磁盘复制到另一端,可以这样 dd if=/dev/sda |ssh root@servername.net "dd of=/dev/sdb", 但是由于ssh是基于加密传输,传输速度会相当慢。但如果使用netcat,基于tcp sockets传输那传输数据。下面是有关基于ssh,nc压缩与不压缩数据的测试。传输的为10G的分区。

\begin{tabular}{l|cc}
 	& Time Elapsed (Sec)	& Speed (MB/s) \\
Over SSH	& 1787.4	& 6.1 \\
Over Netcat (no compression) &	1622.4 & 	6.6 \\
Over Netcat (bzip compression) &	 889.3	& 12.1 \\
Over Netcat (16M block size + bzip)	 & 490.0 &	21.9 \\
  \hline
& Time Savings (Seconds) & 	Percentage Savings \\
Over SSH &	- &	0\% \\
vs Netcat (no compression)	& 165.0 &	9\% \\
vs Netcat (bzip compression)	 & 898.1 &	50\% \\
vs Netcat (16M block size + bzip) &	1297.3	& 73\% 
\end{tabular}


在服务端开始nc服务\textsl{nc -l 19000|bzip2 -d|dd bs=16M of=/dev/sdb}  在客户端开始向服务端传输数据 \textsl{dd bs=16M if=/dev/sda|bzip2 -c|nc serverB.example.net 19000}


备份系统 sudo rsync -aAXv / --exclude={"/dev/*","/proc/*","/sys/*","/tmp/*","/run/*","/mnt/*","/media/*","/lost+found"} root@servername.net:/
dd if=/dev/sda bs=1 count=2048|ssh root@servername.net "dd bs=1 of=/dev/sda"


\section{磁盘扩容}

有时候在针对磁盘分区的时会故意留下一部分空白分区,以供后来不同用途进行再分区挂载。但是当根分区磁盘不够用时,需要把空白分区容量部分分配给根,鉴于此可以使用下面方法扩展根分区。

操作步骤
以CentOS 6.5 64bit 50GB系统盘为例,root分区在最末尾分区(e.g: /dev/xvda1: swap,/dev/xvda2: root)的扩容场景。

执行以下命令,查询当前弹性云服务器的分区情况。

\begin{lstlisting}
[root@sluo-ecs-5e7d ~]# parted -l /dev/xvda
Model: Xen Virtual Block Device (xvd)
Disk /dev/xvda: 53.7GB
Sector size (logical/physical): 512B/512B
Partition Table: msdos

Number  Start   End     Size    Type     File system     Flags
 1      1049kB  4296MB  4295MB  primary  linux-swap(v1)
 2      4296MB  42.9GB  38.7GB  primary  ext4            boot
[root@sluo-ecs-5e7d ~]# blkid

/dev/xvda1: UUID="25ec3bdb-ba24-4561-bcdc-802edf42b85f" TYPE="swap" 
/dev/xvda2: UUID="1a1ce4de-e56a-4e1f-864d-31b7d9dfb547" TYPE="ext4" 
\end{lstlisting}

安装growpart工具。yum install cloud-utils-growpart. 
工具growpart可能集成在cloud-utils-growpart/cloud-utils/cloud-initramfs-tools/cloud-init包里,可以直接执行命令yum install cloud-*确保growpart命令可用即可。


执行以下命令,使用工具growpart将第二分区的根分区进行扩容。

\begin{lstlisting}{bash}
[root@sluo-ecs-5e7d ~]# growpart /dev/xvda 2
CHANGED: partition=2 start=8390656 old: size=75495424 end=83886080 new: size=96465599,end=104856255
# 执行以下命令,检查在线扩容是否成功。
[root@sluo-ecs-5e7d ~]# parted -l /dev/xvda
Model: Xen Virtual Block Device (xvd)
Disk /dev/xvda: 53.7GB
Sector size (logical/physical): 512B/512B
Partition Table: msdos

Number  Start   End     Size    Type     File system     Flags
 1      1049kB  4296MB  4295MB  primary  linux-swap(v1)
 2      4296MB  53.7GB  49.4GB  primary  ext4            boot


[root@sluo-ecs-a611 ~]# resize2fs -f /dev/xvda2
resize2fs 1.42.9 (28-Dec-2013)
Filesystem at /dev/xvda2 is mounted on /; on-line resizing required
old_desc_blocks = 3, new_desc_blocks = 3
\end{lstlisting}



\chapter{网络基础知识}

一个hub就是一个冲突域,同一时间只能有一台pc在发送数据包,其他pc只能监听网络,直到网络空载 csma-cd 

交换机:一个端口就是一个冲突域,广播域。

tcp 面向连接的网络协议, 打电话
udp 无连接网络协议,不可靠传输



\chapter{vim语法总结}


批量删除,以批量注释
命令模式下按 ctrl + v  移动光标选中需要修改的行,然后按x 或者 d 删除该内容或者按 I 或 A 进入编辑模式然后增加内容然后按esc退出后便可以看到所选区域都有显示

vim 下在编辑模式下也有自动补全功能便不是tab键,而是 ctrl + n,
ctrl + o 跳到上一次修改的地方
\chapter{keepalive}

LVS是Linux Virtual Server的简称,也就是Linux虚拟服务器, 是一个由章文嵩博士发起的自由软件项目,它的官方站点是www.linuxvirtualserver.org。现在LVS已经是 Linux标准内核的一部分,在Linux2.4内核以前,使用LVS时必须要重新编译内核以支持LVS功能模块,但是从Linux2.4内核以后,已经完全内置了LVS的各个功能模块,无需给内核打任何补丁,可以直接使用LVS提供的各种功能。

(1)LVS是四层负载均衡,也就是说建立在OSI模型的第四层——传输层之上,传输层上有我们熟悉的TCP/UDP,LVS支持TCP/UDP的负载均衡。因为LVS是四层负载均衡,因此它相对于其它高层负载均衡的解决办法,比如DNS域名轮流解析、应用层负载的调度、客户端的调度等,它的效率是非常高的。
   
(2)LVS的转发主要通过修改IP地址(NAT模式,分为源地址修改SNAT和目标地址修改DNAT)、修改目标MAC(DR模式)来实现。
   
①NAT模式:网络地址转换


\part{存储与数据}
\input{redis/redis.tex}
\part{虚拟化}
\chapter{虚拟化与kvm}
KVM 全称是 基于内核的虚拟机(Kernel-based Virtual Machine),它是Linux 的一个内核模块,该内核模块使得 Linux 变成了一个 Hypervisor:
KVM 是基于虚拟化扩展(Intel VT 或者 AMD-V)的 X86 硬件的开源的 Linux 原生的全虚拟化解决方案。

KVM 中,虚拟机被实现为常规的 Linux 进程,由标准 Linux 调度程序进行调度;
虚机的每个虚拟 CPU 被实现为一个常规的 Linux 线程。这使得 KMV 能够使用 Linux 内核的已有功能。
  但是,KVM 本身不执行任何硬件模拟,需要用户空间程序通过 /dev/kvm 接口设置一个客户机虚拟服务器的地址空间,
  向它提供模拟 I/O,并将它的视频显示映射回宿主的显示屏。目前这个应用程序是 QEMU。
  
\section{虚拟化}
X86操作系统是设计在直接运行在裸硬件设备上的,因此它自动认为完全占有计算机硬件,
X86平台的指令集权限划分为特权模式: Ring0, Ring1, Ring2, Ring3。 操作系统使用Ring0级别;应用程序使用Ring3级别。
驱动程序使用Ring1,Ring2级别。应用程序不能做受控操作,如果需要做,比如要访问硬盘,写文件,那就要通过执行系统调用,执行系统调用的时候CPU的运行级别发生3到0的切换,
并跳转到系统调用对应的内核代码位置执行,内核就为你完成设备访问,完成之后再从0返回3,这个过程也称为用户态的内核态的切换。
X86平台在虚拟化方面的一个难点就是如何将虚拟机越级的指令使用进行隔离。

\begin{figure}[!ht]
    \centering
     \caption{\label{Fig:virtualization-progress} virtualization progress}
    \includegraphics[width=0.8\textwidth]{kvm/virtualization-progress.jpg}
\end{figure}



软件虚拟化、基于二进制翻译的全虚拟化(full Virtualization with Binary Translation):
客户操作系统运行在Ring1,它运行特权指令时,会触发异常(CPU机制,没有权限的指令会触发异常)然后VMM (Virtual Machine Manager)捕获这个异常,在异常里面做翻译,模拟,最后返回到客户操作系统内,客户操作系统认为自己特权指令工作正常,继续运行。但这个性能损耗大。 典型厂商 VMware

\begin{lstlisting}
yum install qemu-kvm qemu-img
yum install virt-install  bridge-utils libvirt
systemctl start libvirtd

qemu-img create -f qcow2 centos-7.1.qcow2 50g
virt-install --name Centos-7.1-x86_64-base  --virt-type kvm \
 --memory 1024 --vcpus 1 \
 --cdrom /data/iso/CentOS-7-x86_64-DVD-1503-01_2.iso \
 --disk path=/base_image/centos-7.1.qcow2,bus=virtio \
 --network network=br0,model=virtio \
 --graphics vnc,listen=0.0.0.0 --noautoconsole
\end{lstlisting}

下面有一个坑, network=br0, 会去找libvirt里network里定义的网络,如果使用bridge,则需要把network=br0 改为bridge=br0既可
如果使用现有qcow2做磁盘,就不需要使用cdrom,直接使用--boot hd既可

创建windows
\begin{lstlisting}
virt-install --name zbddzx_ceshi-05  --ram 8192 --cpus 2 \
 --cdrom=/Data/Base_images/2012R2.iso  \
 --disk path=/opt/zbddzx_ceshi-05.qcow2,bus=virtio  \
 --graphics vnc,listen=0.0.0.0  \
 --network bridge=virbr300,model=virtio \
 --noautoconsole
\end{lstlisting}

guestfish 里面包含很多非常有用的工具,比如virt-copy-in可以把宿主机的文件直接copy进主机
\text{	virt-copy-in /etc/selinux/config  -a /Data/Base_image/CentOS-6.6.qcow2  /etc/selinux/}

安装 KVM 后都会发现网络接口里多了一个叫做 virbr0 的虚拟网络接口,一般情况下,虚拟网络接口virbr0用作nat,以允许虚拟机访问网络服务,但nat一般不用于生产环境。我们可以使用以下方法删除virbr0

1、先使用virsh net-list查看所有的虚拟网络:virsh net-list 
2、卸载与删除virbr0虚拟网络接口 先关闭virsh net-destroy default  再删除virsh net-undefine default

从一个xml文件定义default网络,执行如下命令: \text{virsh net-define /var/lib/libvirt/network/default.xml  }

1、设置virbr0自动启动,执行如下命令:virsh net-start default           

2、自启动 virsh net-autostart default    



https://github.com/webdevops/Dockerfile.git
\chapter{docker基础到kubernetes集群}

\section{docker基础}

在gpu-docker使用
https://github.com/NVIDIA/nvidia-docker

\section{dockerfile 使用总结}

使用dockerfile 心得体会

\subsection{copy 和 add的区别}

COPY <src> <dest>

add 和 copy 都会把 src下面的复制到镜像中。例如 \textbf{COPY data /tmp } 便会把data下面的文件copy到/tmp下,
如果想把整个目录都复制过去那么就必须写成\textbf{COPY data /tmp/data } 这里copy和 add是一样的

虽然他们两功能非常像,在官方文档中的best practices for writing dockerfile时还是推荐使用copy

Although ADD and COPY are functionally similar, generally speaking, COPY is preferred. That’s because it’s more transparent than ADD. COPY only supports the basic copying of local files into the container, while ADD has some features (like local-only tar extraction and remote URL support) that are not immediately obvious. Consequently, the best use for ADD is local tar file auto-extraction into the image,

Because image size matters, using ADD to fetch packages from remote URLs is strongly discouraged; you should use curl or wget instead. That way you can delete the files you no longer need after they’ve been extracted and you won’t have to add another layer in your image.



\subsection{ENTRYPOINT 和 CMD}

docker并不会快照运行的进程,所以通过RUN命令运行的命令仅在 \textbf{docker build} 阶段的时候运行
如果需要在容器启动的时候运行服务需要使用ENTRYPOINT 和 CMD 来指定,并且这两命令都是放在dockerfile
的最后

并且 docker需要让进程一直处于running状态(前台,类似tail -F),也就是说不能运行在后台模式,不然docker会exit,并不会运行
除非特殊需求之外,一般一个容器只运行一个服务,也有时候需要运行多个服务,这时候可以有两种方法来解决,一是把两个服务
写到同一个shell里,然后运行,另一种便是使用supervisord,supervisord看起来是比较重的。

shell 示例

%\lstinputlisting{./scripts/run.sh}

supervisord 示例

\begin{lstlisting}[language=bash]
FROM ubuntu:latest
RUN apt-get update && apt-get install -y supervisor
RUN mkdir -p /var/log/supervisor
COPY supervisord.conf /etc/supervisor/conf.d/supervisord.conf
COPY my_first_process my_first_process
COPY my_second_process my_second_process
CMD ["/usr/bin/supervisord"]
\end{lstlisting}

ENTRYPOINT 会把docker run IMAGE 之外的所以参数都传给 ENTRYPOINT 执行的命令中。CMD则是完全覆盖
当ENTRYPOINT和CMD同时存在的时候 CMD会做为参数传给ENTRYPOINT。在docker run的时候如果有参数转进来,可以理解为覆盖CMD
然后把它做为参数传给ENTRYPOINT.
例如

\begin{lstlisting}[language=bash]
[root@ns1 test]# cat Dockerfile
FROM registry.gsandow.com:5043/centos
MAINTAINER from www.gsandow.com by sandow <j.k.yulei@gmail.com>

#ENTRYPOINT ["echo","entrypoit"]
CMD ["echo","cmd"]
[root@ns1 test]# docker build -t test .
Sending build context to Docker daemon  8.192kB
Step 1/3 : FROM registry.gsandow.com:5043/centos
 ---> 36540f359ca3
Step 2/3 : MAINTAINER from www.gsandow.com by sandow <j.k.yulei@gmail.com>
 ---> Using cache
 ---> c123904bd244
Step 3/3 : CMD echo cmd
 ---> Running in ea01464d732a
 ---> 3bcdf60c6bda
Removing intermediate container ea01464d732a
Successfully built 3bcdf60c6bda
Successfully tagged test:latest
[root@ns1 test]# docker run --rm test
cmd
[root@ns1 test]# docker run --rm test aaa
caused "exec: \"aaa\": executable file not found in \$PATH"
[root@ns1 test]# docker run --rm test echo aaa
aaa
[root@ns1 test]# cat Dockerfile
FROM registry.gsandow.com:5043/centos
MAINTAINER from www.gsandow.com by sandow <j.k.yulei@gmail.com>

ENTRYPOINT ["echo","entrypoit"]
CMD ["echo","cmd"]
[root@ns1 test]# docker build -t test .
[root@ns1 test]# docker run --rm test
entrypoit echo cmd
[root@ns1 test]# docker run --rm test echo 3
entrypoit echo 3
[root@ns1 test]# docker run --rm test cmd
entrypoit cmd
\end{lstlisting}

\noindent
Both CMD and ENTRYPOINT instructions define what command gets executed when running a container. There are few rules that describe their co-operation.

\smallskip
\begin{itemize}
	\item \small{Dockerfile should specify at least one of CMD or ENTRYPOINT commands.}
	\item \small{ENTRYPOINT should be defined when using the container as an executable.}
	\item \small{CMD should be used as a way of defining default arguments for an ENTRYPOINT command or for executing an ad-hoc command in a container.}
	\item \small{CMD will be overridden when running the container with alternative arguments.}

\end{itemize}

\noindent
The table below shows what command is executed for different ENTRYPOINT / CMD combinations:

\noindent
\begin{table}[h]
\begin{tabular*}{\textwidth}{|p{2.8cm}|p{3cm}|p{4cm}|p{5cm}|}
%\begin{tabular*}{0.7\paperwidth}{|l|l|l|l|}
	\hline
	&	No ENTRYPOINT &	ENTRYPOINT exec_entry p1_entry	& ENTRYPOINT [“exec_entry”, “p1_entry”] \\
	\hline
No CMD	&    error, not allowed & /bin/sh -c exec_entry p1_entry & exec_entry p1_entry \\
	\hline
CMD [“exec_cmd”, “p1_cmd”] & exec_cmd p1_cmd & /bin/sh -c exec_entry p1_entry & exec_entry p1_entry exec_cmd p1_cmd\\
	\hline
CMD [“p1_cmd”, “p2_cmd”] & p1_cmd p2_cmd & /bin/sh -c exec_entry p1_entry & exec_entry p1_entry p1_cmd p2_cmd \\
	\hline
CMD exec_cmd p1_cmd & /bin/sh -c exec_cmd p1_cmd & /bin/sh -c exec_entry p1_entry & exec_entry p1_entry /bin/sh -c exec_cmd p1_cmd \\
	\hline

\end{tabular*}
\end{table}

postgresql 官方例中ENTRYPOINT是这个样子的

\begin{lstlisting}
#!/bin/bash
set -e

if [ "$1" = 'postgres' ]; then
    chown -R postgres "$PGDATA"

    if [ -z "$(ls -A "$PGDATA")" ]; then
        gosu postgres initdb
    fi

    exec gosu postgres "$@"
fi

exec "$@"
\end{lstlisting}

	exec
	
	http://xstarcd.github.io/wiki/shell/exec_redirect.html

su,sudo 经常会需要TTY和信号转发行为,它们在设置和使用上比较。
gosu,便是在特定的用户下运行特定的程序然后退出管道。https://github.com/tianon/gosu

\section{docker 使用}

一般启动方式,以gitlab为例

\begin{lstlisting}[language=bash]
sudo docker run --detach \
    --hostname gitlab.example.com \
    --publish 443:443 --publish 80:80 --publish 22:22 \
    --name gitlab \
    --restart always \
    --volume /srv/gitlab/config:/etc/gitlab \
    --volume /srv/gitlab/logs:/var/log/gitlab \
    --volume /srv/gitlab/data:/var/opt/gitlab \
    gitlab/gitlab-ce:latest
\end{lstlisting}

\section{docker compose}

\begin{lstlisting}
web:
  image: 'gitlab/gitlab-ce:latest'
  restart: always
  hostname: 'gitlab.example.com'
  environment:
    GITLAB_OMNIBUS_CONFIG: |
      external_url 'http://gitlab.example.com:9090'
      gitlab_rails['gitlab_shell_ssh_port'] = 2224
  ports:
    - '9090:9090'
    - '2224:22'
  volumes:
    - '/srv/gitlab/config:/etc/gitlab'
    - '/srv/gitlab/logs:/var/log/gitlab'
    - '/srv/gitlab/data:/var/opt/gitlab'
\end{lstlisting}

docker-compose up -d


\section{参考链接}

\href{https://docs.docker.com/engine/userguide/eng-image/dockerfile_best-practices/}{dockerfile_best-practices}
\href{https://stackoverflow.com/questions/21553353/what-is-the-difference-between-cmd-and-entrypoint-in-a-dockerfile}{what-is-the-difference-between-cmd-and-entrypoint-in-a-dockerfile}
\href{https://stackoverflow.com/questions/24958140/what-is-the-difference-between-the-copy-and-add-commands-in-a-dockerfile}{what-is-the-difference-between-the-copy-and-add-commands-in-a-dockerfile}
\href{https://docs.docker.com/engine/admin/multi-service_container/}{ulti-service_container}


\section{kubernetes 组件及基础概念}

\href{ https://www.youtube.com/watch?v=_vHTaIJm9uY&list=PLF3s2WICJlqOiymMaTLjwwHz-MSVbtJPQ}{kubernetes 基础介绍视频链接}

\subsection{kube-master}

\begin{itemize}

	\item Kubernetes is highly api centered. The Kubernetes API server validates and configures data for the api objects which include pods, services, replicationcontrollers, and others. The API Server services REST operations and provides the frontend to the cluster’s shared state through which all other components interact.

        \item The Kubernetes scheduler is a policy-rich, topology-aware, workload-specific function that significantly impacts availability, performance, and capacity. The scheduler needs to take into account individual and collective resource requirements, quality of service requirements, hardware/software/policy constraints, affinity and anti-affinity specifications, data locality, inter-workload interference, deadlines, and so on. Workload-specific requirements will be exposed through the API as necessary.

        \item The Kubernetes controller manager is a daemon that embeds the core control loops shipped with Kubernetes. In applications of robotics and automation, a control loop is a non-terminating loop that regulates the state of the system. In Kubernetes, a controller is a control loop that watches the shared state of the cluster through the apiserver and makes changes attempting to move the current state towards the desired state. Examples of controllers that ship with Kubernetes today are the replication controller, endpoints controller, namespace controller, and serviceaccounts controller

\end{itemize}

The Kubernetes network proxy runs on each node. This reflects services as defined in the Kubernetes API on each node and can do simple TCP,UDP stream forwarding or round robin TCP,UDP forwarding across a set of backends. Service cluster ips and ports are currently found through Docker-links-compatible environment variables specifying ports opened by the service proxy. There is an optional addon that provides cluster DNS for these cluster IPs. The user must create a service with the apiserver API to configure the proxy.
 
The kubelet is the primary “node agent” that runs on each node. The kubelet works in terms of a PodSpec. A PodSpec is a YAML or JSON object that describes a pod. The kubelet takes a set of PodSpecs that are provided through various mechanisms (primarily through the apiserver) and ensures that the containers described in those PodSpecs are running and healthy. The kubelet doesn’t manage containers which were not created by Kubernetes.

Other than from an PodSpec from the apiserver, there are three ways that a container manifest can be provided to the Kubelet.
File: Path passed as a flag on the command line. Files under this path will be monitored periodically for updates. The monitoring period is 20s by default and is configurable via a flag.
HTTP endpoint: HTTP endpoint passed as a parameter on the command line. This endpoint is checked every 20 seconds (also configurable with a flag).
HTTP server: The kubelet can also listen for HTTP and respond to a simple API (underspec’d currently) to submit a new manifest.

fluentd is the component which is basically responsible for managing the logs and talking to the central locking mechanism

\subsection{kube-node}



\href{https://github.com/kubernetes/kops}{kops   安装,升级,管理工具}


flanneld

\begin{lstlisting}
/usr/lib/sysctl.d/00-system.conf
net.bridge.bridge-nf-call-iptables=1
net.bridge.bridge-nf-call-ip6tables=1
sysctl -w net.ipv4.ip_forward=1
sysctl -w net.bridge.bridge-nf-call-ip6tables=1
sysctl -w net.bridge.bridge-nf-call-iptables=1
\end{lstlisting}

0down vote	What parameters did you provide for kubeadm?
	If you want to use flannel as the pod network, specify --pod-network-cidr 10.244.0.0/16 if you’re using the daemonset manifest below. However, please note that this is not required for any other networks besides Flannel
	Execute these commands on every node:





\subsection{ pod}

pod is a group of one or more containers that are always co-located and
co-scheduled that share the context, containers in a pod share the same
IP address,ports,hostname, and storage. modeled like a virtual machine:
each container represnets one process
tightly coupled with other containers in the same pod

pod are scheduled in nodes, fundamental unit of deployment in kubernetes.

use cases for pod



\subsection{Replication Controller}

Ensures that a pod or homogeneous set of pods are always up and available
Always maintains desired number of pods
if there are exess pod, they get killed
new pods are launched when they fail, get deleted, or terminated

creating a replication controller with a count of 1 ensoures that a pod 
is always availble
RC and Pods are associated througth lables

\subsection{replica set}

replica set is the advancement to replication controller
replica sets are the nest generation Replication controller
ensures specified numbers of pods are always running
Pods are replaced by Replica Sets when a failure occurs
lables and selectors are useed for associating pods with replica set
usually combined with pods when defining the deployment


如果定义的pod,删除或者终止后不会重建,定义成RC后,删除或者终结后还会重建一个

kubectl scale rc web --replicas=20  
也可以这样来定制RC


separate statefull containers and stateless containers because stateful containers have very
stringent requirements for example i might want to have an SSD based storage that is mounted
as a volume 


\subsection{ Service}
a Service is an abstraction of a logical set of Pods defined by a policy
it acts as the intermediary for pods to talk to each other
selectors are used for accesing all the pods that match a specific lable
service is an object in kubernetes - similar to pods and RCs
each Service exposes one of more ports and targetPorts: port will expose to its consumers
the targetPorts is how it is going to route the traffic to the destination pods
The targetPort is mapped to the port exposed by matching Pods
Kubernetes Services Support TCP and UDP portocols


kubectl create -f pod.yml -f rc.yml 
kubectl create -f svc.yml

kubectl apply -f svc.yml


red, green, blue 可以分别定义开发,测试,正式环境,然后通过svc提供服务,只需要改变select便可以轻松切到某个环境

\section{kubectl高级用法}

kubectl 来获取特殊的值

\begin{lstlisting}

获取node name
 kubectl get nodes -o jsonpath='{range.items[*].metadata}{.name} {end}'

获取node ip
kubectl get nodes -o jsonpath='{range .items[*].status.addresses[?(@.type=="ExternalIP")]}{.address} {end}'

 一:get 过滤及格式化输出
kubectl get pods --all-namespaces -o jsonpath="{..image}"
kubectl get pods --all-namespaces -o jsonpath="{.items[*].spec.containers[*].image}"

    * .items[*]: for each returned value
    * .spec: get the spec
    * .containers[*]: for each container
    * .image: get the image
上面的一般都不会格式化输出,需要使用range来结合使用
kubectl get pods --all-namespaces -o=jsonpath='{range .items[*]}{"\n"}{.metadata.name}{":\t"}{range .spec.containers[*]}{.image}{", "}{end}{end}' 

Kubectl run my-web —image=nginx —port=80
Kubectl expose deployment my-web —target-port=80 —type=NodePort

Kubectl get svc my-web -o to-templates=‘{{(index .spec.ports 0).nodePort}}’

切到另一个集群
kubectl config view
kubectl config user-context xxxx
kubectl config use-context  xxxx

\end{lstlisting}




\section{定义pod}

\subsection{限制容器使用资源}

创建包含一个容器的Pod,这个容器申请100M的内存,并且内存限制设置为200M
放在contaner 下面

jkljl \cite{Survey2014}
\printbibliography

\begin{lstlisting}
    resources:
      limits:
        memory: "200Mi"
      requests:
        memory: "100Mi"
\end{lstlisting}


\subsection{pod生命周期与重启策略}

A pod (as in a pod of whales or pea pod) is a group of one or more containers (such as Docker containers), with shared storage/network, and a specification for how to run the containers. A pod’s contents are always co-located and co-scheduled, and run in a shared context. A pod models an application-specific “logical host” - it contains one or more application containers which are relatively tightly coupled — in a pre-container world, they would have executed on the same physical or virtual machine.

\href{https://kubernetes.io/docs/concepts/workloads/pods/pod/#termination-of-pods}{ermination-of-pods}

The kubelet can optionally perform and react to two kinds of probes on running Containers:

The kubelet uses liveness probes to know when to restart a Container. For example, liveness probes could catch a deadlock, where an application is running, but unable to make progress. Restarting a Container in such a state can help to make the application more available despite bugs.

The kubelet uses readiness probes to know when a Container is ready to start accepting traffic. A Pod is considered ready when all of its Containers are ready. One use of this signal is to control which Pods are used as backends for Services. When a Pod is not ready, it is removed from Service load balancers.

\begin{description}
\item[livenessProbe] Indicates whether the Container is running. If the liveness probe fails, the kubelet kills the Container, and the Container is subjected to its restart policy. If a Container does not provide a liveness probe, the default state is Success.
\item[readinessProbe] Indicates whether the Container is ready to service requests. If the readiness probe fails, the endpoints controller removes the Pod’s IP address from the endpoints of all Services that match the Pod. The default state of readiness before the initial delay is Failure. If a Container does not provide a readiness probe, the default state is Success
\end{description}

Probes have a number of fields that you can use to more precisely control the behavior of liveness and readiness checks:

\begin{description}
	\item[initialDelaySeconds] Number of seconds after the container has started before liveness or readiness probes are initiated.
	\item[periodSeconds] How often (in seconds) to perform the probe. Default to 10 seconds. Minimum value is 1.
	\item[timeoutSeconds] Number of seconds after which the probe times out. Defaults to 1 second. Minimum value is 1.
	\item[successThreshold] Minimum consecutive successes for the probe to be considered successful after having failed. Defaults to 1. Must be 1 for liveness. Minimum value is 1.
	\item[failureThreshold] When a Pod starts and the probe fails, Kubernetes will try failureThreshold times before giving up. Giving up in case of liveness probe means restarting the Pod. In case of readiness probe the Pod will be marked Unready. Defaults to 3. Minimum value is 1.
\end{description}

HTTP probes have additional fields that can be set on httpGet:

\begin{description}
	\item[host] Host name to connect to, defaults to the pod IP. You probably want to set “Host” in httpHeaders instead.
	\item[scheme] Scheme to use for connecting to the host (HTTP or HTTPS). Defaults to HTTP.
	\item[path] Path to access on the HTTP server.
	\item[httpHeaders] Custom headers to set in the request. HTTP allows repeated headers.
	\item[port] Name or number of the port to access on the container. Number must be in the range 1 to 65535.
For an HTTP probe, the kubelet sends an HTTP request to the specified path and port to perform the check. The kubelet sends the probe to the pod’s IP address, unless the address is overridden by the optional hostfield in httpGet. If scheme field is set to HTTPS, the kubelet sends an HTTPS request skipping the certificate verification. In most scenarios, you do not want to set the host field. Here’s one scenario where you would set it. Suppose the Container listens on 127.0.0.1 and the Pod’s hostNetwork field is true. Then host, under httpGet, should be set to 127.0.0.1. If your pod relies on virtual hosts, which is probably the more common case, you should not use host, but rather set the Host header in httpHeaders.
\end{description}

\href{https://kubernetes.io/docs/tasks/configure-pod-container/configure-liveness-readiness-probes/}{configure-liveness-readiness-probes}

A Probe is a diagnostic performed periodically by the kubelet on a Container. To perform a diagnostic, the kubelet calls a Handler implemented by the Container. There are three types of handlers:

\begin{description}
\item[ExecAction] Executes a specified command inside the Container. The diagnostic is considered successful if the command exits with a status code of 0.
\item[TCPSocketAction] Performs a TCP check against the Container’s IP address on a specified port. The diagnostic is considered successful if the port is open.
\item[HTTPGetAction] Performs an HTTP Get request against the Container’s IP address on a specified port and path. The diagnostic is considered successful if the response has a status code greater than or equal to 200 and less than 400.
\end{description}

Each probe has one of three results:
Success: The Container passed the diagnostic.
Failure: The Container failed the diagnostic.
Unknown: The diagnostic failed, so no action should be taken.

A PodSpec has a restartPolicy field with possible values Always, OnFailure, and Never. The default value is Always. restartPolicy applies to all Containers in the Pod. restartPolicy only refers to restarts of the Containers by the kubelet on the same node. Failed Containers that are restarted by the kubelet are restarted with an exponential back-off delay (10s, 20s, 40s …) capped at five minutes, and is reset after ten minutes of successful execution. As discussed in the Pods document, once bound to a node, a Pod will never be rebound to another node.


\href{https://kubernetes.io/docs/concepts/workloads/pods/pod-lifecycle/}{pod-lifecycle}

Kubernetes supports the postStart and preStop events. Kubernetes sends the postStart event immediately after a Container is started, and it sends the preStop event immediately before the Container is terminated

https://kubernetes.io/docs/tasks/configure-pod-container/attach-handler-lifecycle-event/
https://kubernetes.io/docs/concepts/containers/container-lifecycle-hooks/

\subsection{创建pod拉取私有库镜像}
拉取镜像由 imagePullPolicy 来控制拉取策略,他有三个值   Always IfNotPresent Never


当使用私有库拉镜像的时候需要创建secret 

kubectl create secret docker-registry myregistrykey --docker-server=DOCKER_REGISTRY_SERVER --docker-username=DOCKER_USER --docker-password=DOCKER_PASSWORD --docker-email=DOCKER_EMAIL
secret "myregistrykey" created.

使用yaml,来创建secret这时候有一点麻烦,需要先docker login 登录私钥库,可以看到在家目录多出来个.docker隐藏目录,下面有config.json文件,这时需要使用base64加密,在定义secrete 时需要把加密后的值连续的赋予给data[".dockerconfigjson"]

\begin{lstlisting}
docker login  -u admin -p Harbor123  10.10.39.226
[root@mobius_004 ~]# cd .docker/
[root@mobius_004 .docker]# ls
config.json
[root@mobius_004 .docker]# cat config.json
{
	"auths": {
		"10.10.39.226": {
			"auth": "YWRtaW46SGFyYm9yMTIzNDU="
		}
	},
	"HttpHeaders": {
		"User-Agent": "Docker-Client/18.01.0-ce (linux)"
	}
}
[root@mobius_004 .docker]# base64EncodeData=$(base64 -w 0 config.json)
[root@mobius_004 .docker]# echo $base64EncodeData|base64 --decode
apiVersion: v1
kind: Secret
metadata:
  name: myregistrykey
  namespace: awesomeapps
data:
  .dockerconfigjson: $base64EncodeData

type: kubernetes.io/dockerconfigjson

然后在创建pod的时候指定secrets来拉镜像

apiVersion: v1
kind: Pod
metadata:
  name: foo
  namespace: awesomeapps
spec:
  containers:
    - name: foo
      image: janedoe/awesomeapp:v1
  imagePullSecrets:
    - name: myregistrykey

\end{lstlisting}

https://kubernetes.io/docs/concepts/containers/images/

\section{kubeconfig 配置}

kubeconfig 在kubectl, kubelet, kube-proxy,bootstrap都会用到,配置kubeconfig可以用三种方式

通过命令方式

\begin{lstlisting}
 kubectl config set-cluster kubernetes \
  --certificate-authority=/etc/kubernetes/ssl/ca.pem \
  --embed-certs=true \
  --server=${KUBE_APISERVER} \
  --kubeconfig=kube-proxy.kubeconfig
 # 设置客户端认证参数
 kubectl config set-credentials kube-proxy \
  --client-certificate=/etc/kubernetes/ssl/kube-proxy.pem \
  --client-key=/etc/kubernetes/ssl/kube-proxy-key.pem \
  --embed-certs=true \
  --kubeconfig=kube-proxy.kubeconfig
 # 设置上下文参数
 kubectl config set-context default \
  --cluster=kubernetes \
  --user=kube-proxy \
  --kubeconfig=kube-proxy.kubeconfig
 # 设置默认上下文
 kubectl config use-context default --kubeconfig=kube-proxy.kubeconfig
 mv kube-proxy.kubeconfig /etc/kubernetes/
\end{lstlisting}

直接编辑方式

.kube/config 这个配置文件可以指定文件或者指定base64值


\begin{lstlisting}
apiVersion: v1
kind: Config
users:
- name: kubelet
  user:
    client-certificate-data: <base64-encoded-cert>
    client-key-data: <base64-encoded-key>
clusters:
- name: local
  cluster:
    certificate-authority-data: <base64-encoded-ca-cert>
contexts:
- context:
    cluster: local
    user: kubelet
  name: service-account-context
current-context: service-account-context

\end{lstlisting}

加密密钥可以使用 base64 /Users/yulei/Documents/ansible/roles/kubernetes/files/ssl/ca.pem 便可得到

To generate the base64 encoded client cert, you should be able to run something like cat /var/run/kubernetes/kubelet_36kr.pem | base64. If you don't have the CA certificate handy, you can replace the certificate-authority-data: <base64-encoded-ca-cert> with insecure-skip-tls-verify: true.

If you put this file at /var/lib/kubelet/kubeconfig it should get picked up automatically. Otherwise, you can use the --kubeconfig argument to specify a custom location.

或者在config里指定文件

apiVersion: v1
clusters:
- cluster:
    certificate-authority: /etc/kubernetes/certs/ca.crt
    server: https://kubernetesmaster
  name: default-cluster
contexts:
- context:
    cluster: default-cluster
    user: default-admin
  name: default-system
current-context: default-system
kind: Config
preferences: {}
users:
- name: default-admin
  user:
    client-certificate: /etc/kubernetes/certs/server.crt
    client-key: /etc/kubernetes/certs/server.key


\section{生产环境中使用kubernetes}

Kubernetes in prod

https://techbeacon.com/one-year-using-kubernetes-production-lessons-learned

https://github.com/kelseyhightower/confd

https://www.graylog.org/

https://www.loggly.com/blog/top-5-docker-logging-methods-to-fit-your-container-deployment-strategy/

https://medium.com/readme-mic/kubernetes-1-year-in-production-f406bdb95c22

https://kubernetes.io/docs/reference/generated/kubernetes-api/v1.9/


https://blog.dockbit.com/kubernetes-canary-deployments-for-mere-mortals-6696910a52b2

https://kubernetes.io/docs/concepts/overview/what-is-kubernetes/

https://www.loggly.com/resource/log-management-handbook-docker/

https://medium.com/readme-mic/kubernetes-1-year-in-production-f406bdb95c22


https://medium.com/readme-mic/kubernetes-1-year-in-production-f406bdb95c22

\subsection{openshift}

https://github.com/openshift/origin

http://www.linkedin.com/pulse/part-2-kubernetes-services-minikube-docker-james-denman


学习集群

https://www.katacoda.com/courses/kubernetes/playground


\section{kubernetes addon}


\subsection{Discovering Services}
discovering services - dns
the DNS server watches kubernetes API for new Services
the DNS server creates a set of DNS recordes for each Services
Services can be resolved by the name within the same namespace
Pods in other namespaces can access the Service by adding the namespace to the DNS path
  my-service.my-namespace

Discovering Services - env vars

Service Types
 ClusterIP: service is reachable only from inside of the cluster
NodePort
 Service is reachable through NodeIP:NodePort address
LoadBalancer
  service is reachable through an external load balancer mapped to NoderIP:NodePort address

kubernetes create Docker link compatible environment variables in all pods
containers can use the environment variable to talk to the service endpoint

https://segmentfault.com/a/1190000002892825


\section{pod的持久化}

persistence in pods

pods are ephemeral and stateless

volumes bring persistence to pods

kubernetes voluems are similar to docker volumes, but managed differently

all containers in pod can access the volumes

volumes are associated with the lifecycle of pod

directories in the host are exposed as volumes

volumes may be based on a variety of storage backends

kubernetes have three method to persistence into your workload

1:  basic volume will persistence to you pod with from limitations
2: rely on distributed storage like NFS
3: clear dispersing,

kubernetes volume types

\begin{lstlisting}
- hostbased
   emptydir
    hostpath
- block storage
  amazon EBS
  GCE Psersistent Disk
  Azure Disk
  VSphere Volume
- Distributed file System
  NFS
  Ceph
  Gluster
  Amazon EFS
- other
  Flocker
  iScsi
  Git Repo



gcloud container clusters get-credentials jani-gke-demo asia-east1-a
gcloud compute disks create --size=10G --zone=asia-east1-a my-data-disk
gcloud compute disks delete --zone=asia-east1-a my-data-dis

\end{lstlisting}


Unserstanding Psersistent Volume and Claims

PersistentVolume(PV)
  Networked storage in ther cluster per-provisioned  by a administrator
PersistentVolumeClaim (PVC)
  Storage resource requested by a user.
StorageClass
  types of supported storage profiles offered by administrator




\part{持续集成与自动化}
\chapter{自动化管理工具}
服务器环境中要想保证其稳定运行,必不可少的便是标准化,自动化,设想任何一个运维人员都是上去手动修改主机配信息,一旦出故障,如果此运维人员还在职,且还记得修改过什么配置,还可以恢复回来,但恢复时长也相当长,这对IT管理造成相当大的困难,公司服务器标准化,自动化势在必行。
\section{ansible}
Ansible 基于python研发的自动化运维工具, ansible是无客户端也不需要启服务端工具,十分方便,主要基于openssl所以安全性也比较高,但是因为任务按队列依次执行,所以并没有saltsatck那样并发的快.特别是维护上百台机子后会感觉到明显慢很多。
\begin{itemize}
\item ansible core : ansible 自身核心模块
\item host inventory: 主机库,定义可管控的主机列表
\item connection plugins: 连接插件,一般默认基于 ssh 协议连接
\item modules:core modules ( 自带模块 ) 、 custom modules ( 自定义模块 )
\item playbooks :剧本,按照所设定编排的顺序执行完成安排任务
\end{itemize}
我们可以使用fetch模块来收集配置文件,在play book里不仅可以指定vars变量,还可以指定vars 文件, var files
\begin{lstlisting}
- hosts: myhosts
  vars_files:
    - default_step.yml
\end{lstlisting}

ansible 经常会配置文件,如果配置变更则需要重启服务,此时需要使用notify模块

\section{salt}

\subsection{salt介绍}
与ansible不同的是,salt是一个C/S架构的软件,salt管理端为master,客户端叫minion,通过server端下发指令,客户端受指令的方式进行操作,saltstack基于zeromq消息队列来管理成千上万台主机客户端,传输指令执行相关操作。采用RSA key方式进行身份确认,传输采用AES方式进行加密,这使得它的安全性得到了保证。

在每个minion启动后便会自动生成RSA公密钥,存入于/etc/salt/pki/minion,中,根据minion配置文件中master地址,主动发送公钥给master等待master接收,master接收后便可以批量管理主机。

\subsection{salt中grains与pillar}

Grains 是saltstack组件之一,记录saltstack Minion 的一些静态信息的组件,(CPU, 内存, 磁盘, 网络, 等) 可以通过grains.items查看某台minion的所有Grain信息,minion的grains信息会在minions启动时汇报给master,在实际应用环境中我们需要根据自己的业务需求去算定义grains,在每次修改完grains后需要同步更新grains.  salt '*' saltutil.sync_grains。 了解更多关于grains函数使用命令查看 salt -E 'client*' sys.list_functions grains

自定义grians有三种方法: 
第一种在/etc/salt/master里直接配置,

\begin{lstlisting}
grains:
  roles:
    - webserver
    - memcache
\end{lstlisting}

第二种在另起一个文件/etc/salt/grains在里面定义,

\begin{lstlisting}
roles:
  - webserver
  - memcache
\end{lstlisting}

第三种使用python定义在minion配置文件中配置grains 放到任何环境中_grains目录下

\begin{lstlisting}
[root@linux-node1 /srv/salt/_grains]# cat my_grains.py
#!/usr/bin/env python
#-*- coding: utf-8 -*-

def my_grains():
    # 初始化一个grains字典
    grains = {}
    grains['iaas'] = 'openstack'
    grains['edu'] = 'sandow'
    return grains
[root@linux-node1 /srv/salt/_grains]# cat roles.py
#!/usr/bin/env python
#-*- coding: utf-8 -*-
import os.path
def roles():
    roles_file= "/etc/salt/roles"
    roles_list= []
    if os.path.isfile(roles_file):
        roles_fd = open(roles_file, "r")
        for eachroles in roles_fd:
            roles_list.append(eachroles[:-1])
    return {'roles': roles_list}
if __name__ == "__main__":
    print roles()
\end{lstlisting}

三种方法优先级为从高到低依次为系统自带,grains文件配置,master grains. 
\subsubsection{Pillar}

% TODO pillar还没有好好学习整理,不知道下次又是啥时候整理了
数据管理中心 Pillar
Pillar 也是salt组件之一,叫数据管理中心,或者说是配置管理中心。 会经常配合states在大规模配置管理工作中使用它,pillar在saltstack中主要的作用就是存储和定义配置管理中需要的一些数据,比如软件版本号,用户名,密码,配置等信息,它的定义存储格式跟grains类似,同样增加完pillar配置后需要刷新 salt '*' saltutil.refresh_pillar。 查看pillar salt '*' pillar.items 
 
master 端配置文件中指定了pillar的文件存放位置,
\begin{lstlisting}
pillar_roots:
 base:
 - /srv/pillar
\end{lstlisting}

同状态模块一样,里面需要有top.sls指定入口文件,编写方式也一样。



grains 与 pillar的区别
名称	存储位置	数据类型	数据采集更新方式	应用
Grains	Minion端	静态数据	Minion启动时收集,也可以使用saltutil.sync_grans进行刷新	存储Mnion基本数据,比如用于匹配Minion,自身数据可以用来做资产管理等。
Pillar	Master端	动态数据	在master端定义,指定给对应的minion,可以使用saltutil.refresh_pillar刷新	存储master指定的数据,只有指定的Minion可以看到。用于敏感数据保存

\subsection{远程执行}

% TODO Target没有细化,salt命令没有讲清 
%运程执行本意是讲 salt这个命令怎么用

有时候仅需要使用salt运行简单的命令,可以使用cmd.run模块  \textbf{salt TARGET cmd.run 'w'}
目标端 target 指定主机名,这里可以使用到正则匹配,grains匹配,pillar匹配,主要组,或者直接列出主机名,这里的匹配在top.sls也可以同样适用

目标可以通过正则来匹配minion id. 或者用grains, pillar subnet/ip address, compound matching, node groups 来匹配


管理对象 target
saltstack系统中我们的管理对象叫作target,在master上我们可以采用不同的target 去管理不同的minion
在target options下可以分很多种匹配方式

分发文件
1. salt-cp 批量分发文件
salt-cp 语法格式为
salt-cp '*' [ options ] SOURCE DEST


\subsection{jinja}


%TODO jinja 应该再细化一下
在编写状态文件的时候经常会引用变量,grains, pillar,这时候就需要使用jinja

变量使用Grains: {{ grains[‘fqdn_ip4’] }}

变量使用执行模块: {{ salt['network.hw_addr']('eth0') }}

变量使用Pillar: {{ pillar[‘apache'][‘PORT'] }}

jinja 模版来写keepalived的优写级

\begin{lstlisting}

- ROUTEID: HAPROXY_MASTER
- STATEID: MASTER
- PRIORITYID: 101

- ROUTEID: HAPROXY_BACKUP
- STATEID: BACKUP
- PRIORITYID: 100

\end{lstlisting}

\begin{lstlisting}


  



{{ motdfile }}:
  file.managed:
    - source: salt://motd

\end{lstlisting}

\subsection{状态模块 state}

状态模块描述minion端的状态,按照官网的说明,往往最强大,最有用的工程解决方案都是基于简单的原则,
(Many of the most powerful and useful engineering solutions are founded on simple principles. Salt States strive to do just that: K.I.S.S. (Keep It Stupidly Simple))

salt state的核心便是sls文件(salt state file) sys文件描述了那些系统应该是什么样子。 sys是以yaml为格式序列化存储数据,所以其本质上就是字典,列表,数字,举个例子

\begin{lstlisting}
apache:
  pkg.installed: []
  service.running:
    - enable: True
    - require:
      - pkg: apache
\end{lstlisting}

这个sls状态文件将会确保apache已经安装,并且已经在运行。第一行apache是这个数据集的ID,全局惟一,一个ID下可以有多个模块,但是不能使用多次使用同一个模块。第二三行表示那些状态模块需要运行。基本模式是\textbf{<state_module>.<function>}, pkg.installed 确定当前主机已经安装了指定软件,如果不指定pkgs则默认安装第一行ID名。第三行service.running表示确保软件已经在运行。如果不指定name,默认以ID为软件名。
最后两行require表示service.running需要依赖于ID为apache下的pkg模块运行完后才会执行。 所以上面可以修改为

\begin{lstlisting}
testpkg:
  pkg.installed:
    - pkgs:
      - httpd
  service.running:
    - name: httpd
    - enable: True
    - require:
      - pkg: testpkg
\end{lstlisting}


要想运行状态文件需要在/etc/salt/master中开始file_roots配置
\begin{lstlisting}
file_roots:
  base:
    - /srv/salt
  dev:
    - /srv/salt/dev/services
    - /srv/salt/dev/states
  prod:
    - /srv/salt/prod/services
    - /srv/salt/prod/states
\end{lstlisting}

base, dev, prod表示环境,salt默认会去base环境下去找状态文件,假设把上面内容保存到/srv/salt/apache/init.sls, 要运行单个sls文件可以使用命令\textbf{salt '*' state.sls apache}运行,这里的apache,salt会去base环境下找apache.sls,如果没有,会继续找有没有目录apache,并且下面有init.sls,如果都没有则返回错误。如果要运行/srv/salt/apache/install.sys 最后的状态文件变成  apache.install 既可



当apache目录下有多个sls时,可以使用include apache.xxx 来引到当前文件中

但是如果环境比较多,不同的主机运行不同的状态文件,你又不想一次次敲命令,又乱又容易弄错怎么办,这时候就出现top.sls,在整个salt状态文件里惟一,他定义了针对不同环境下不同主机运行不同的状态文件。默认放到base环境根目录下,也就是/srv/salt下。
\begin{lstlisting}
base:
  'os:Fedora':
    - webserver
    - match: grain
dev:
  'dev-*':
    - vim
  'db*dev*':
    - db
prod:
  '10.10.200.0/24':
    - match: ipcidr
    - deployments.qa.site1
\end{lstlisting}

最后使用命令\textbf{salt '*' state.highstate},一次搞定。

这里仅列出简单两个模块两个方法的用法,salt有非常多的模块可以使用,可以通过命令来获取所有模块以及模块中的方法及用途

\begin{description}
\item{查看state模块} 'Minion' sys.list_state_modules
\item{查看指定states(git)的所有functions} salt 'Minion' sys.list_state_functions git 
\item{查看指定function的用法} salt 'Minion' sys.state_doc git.config
\end{description}



针对管理对象操作
module 是我们日常使用saltstack最多的一个组件,是用于管理对象操作的,这也是saltstack通过push的方式管理的入口,比如管理日常简单的执行命令,查看包安装情况,查看服务运行情况等都是通过 module来实现的 


为什么要加require?就是因为salt本身是并发的去处理任务,service和pkg有可能现时运行,这样有可能达不到预期结果,所以加着require做前后依赖。除require外,还有很多条件

require: 在执行这一步之前需要满足的东西都列在下面,不满足就不执行,可以依赖整个sls, {'require': [{"sls":"foo"},]},require_in 反过来被谁依赖

watch 里面任何一个状态变化变触发,并不是所有的state都支持watch, service state能支持,watch_in被谁监控

unless执行下面内容,如果结果为False才执行该state, {"unless":["rpm -q vim-enhanced","ls /usr/bin/vim"]}, onlyif 结果返回为True才执行该state

 
onfail, 另一个state执行失败后执行这个, onchanges另一个state执行成功并且产生变化执行,prereq 被要求在xxx state之前执行, use 利用另一个state的参数,
listen/listen_in 和watch/watch_in 类似在所有state最后执行, include 组合多个state, extend 对之前内容扩展, 


\subsection{salt实践}

环境准备,使用两台主机做测试,并且在每一台机子都做hosts解析 

\begin{itemize}
\item master 端 主机名master_101 ip 172.16.1.101
\item minion 端 主机名client_102 ip 172.16.1.102
\end{itemize}

\begin{lstlisting}
#安装软件
yum install salt-master salt-minion -y

#master端
systemctl  start salt-master
systemctl  enable salt-master 

# minion端
sed -i 's/#master: salt/master: 172.16.1.101/g' /etc/salt/minion
systemctl  start salt-minion 
systemctl  enable salt-minion 
# 接收 公钥
salt-key -a client_102 -y
 Accepted Keys:
 client_102
 Denied Keys:
 Unaccepted Keys:
 Rejected Keys:
#测试连通性
salt 'client_102' test.ping
client_102:
True
\end{lstlisting}

使用salt-api



\part{监控与数据展示}
\input{zabbix/zabbix.tex}
\chapter{elk 简单介绍}
\section{elasticsearch}
elasticsearch其功能主要用于存储与搜索
curl-XGET 'http://localhost:9200/_nodes'

删除日志, curl -XDELETE 127.0.0.1:9200/fudao_requestlog-2017.12.10

列出所有索引并显示状态与索引大小。 curl -XGET 10.9.199.212:9200/_cat/indices 
\section{logstash}
logstash 是一个轻量,开源的服务端数据处理管道,能够收集来自各种来源的数据,实时转换并将其发送到目标位置。其包含三个部分,input,filter,output,下面分三部分一一介绍


\subsection{输入input}
采集各种样式、大小和来源的数据,数据往往以各种各样的形式,或分散或集中地存在于很多系统中。 Logstash 支持各种输入选择 ,可以在同一时间从众多常用来源捕捉事件,\href{https://www.elastic.co/guide/en/logstash/current/input-plugins.html}{支持的输入plugins},里面有各种详细介绍,这里就简单介绍一个用redis做为input的例子

\begin{lstlisting}
	input {
    redis {
        data_type => "pattern_channel"
        key => "logstash-*"
        host => "192.168.0.2"
        port => 6379
        threads => 5
        type => "redis-test"
    }
}

\end{lstlisting}

\subsection{过滤 filter}
数据从源传输到存储库的过程中,Logstash 过滤器能够解析各个事件,识别已命名的字段以构建结构,并将它们转换成通用格式,以便更轻松、更快速地分析和实现商业价值。利用grok从非结构化数据中派生了结构,从IP地址破译出地理坐标,\href{https://www.elastic.co/guide/en/logstash/current/filter-plugins.html}{filter plugins}

针对不同来源,不同类型数据做不同处理及输出,可以在input中增加type 这个字段,然后在filter,output的时候使用type这个字段来判断来源是什么,怎么处理,怎么输出。也可以使用input过来默认增加的字段来处理。

在logstash里字段都需要使用 [fieldname]来进行处理,字段分为top-level(agent, ip, request, ua, response), nested filed(status, bytes, os)。在指定nested filed的时候可以[ua][os]
\begin{lstlisting}
	filter{
     if[source]=~"ftp.log"{
        grok{
         match=>{
                 "message"=>[
                 "\[%{TIMESTAMP_ISO8601:timestamp}\] ALL AUDIT: User \[%{GREEDYDATA:userId}\]\ %{GREEDYDATA:var} \[%{HOSTNAME:ip}\] %{GREEDYDATA:event}.",
                 "\[%{TIMESTAMP_ISO8601:timestamp}\] ALL AUDIT: User \[%{GREEDYDATA:userId}\]\ %{GREEDYDATA:event} \[%{GREEDYDATA:filename}\]."
							]
				}
				
				
				if[event]=~"retrieving file"{
						add_tag=>["Download"]
					}else if["event"]=~"storing file"{
						add_tag=>["Upload"]
					}else if["event"]=~"has logged in"{
						add_tag=>["Login"]			
				}
				add_tag=>["log_ftp"]
			}
			
			}
     }
\end{lstlisting}

在处理数据的时候经常会用到if判断,判断的表达式有, ==, !=, <, >, <=, >= ,正常 =~, !~, in, not in, and, or,nand, xor,   !意思是否定的意思, 多个表达试可以使用()括起来。
\begin{lstlisting}

if EXPRESSION {
  ...
} else if EXPRESSION {
  ...
} else {
  ...
}
	
\end{lstlisting}

使用mutate来移除一个字段
\begin{lstlisting}
filter {
  if [action] == "login" {
    mutate { remove_field => "secret" }
  }
}
\end{lstlisting}

使用in判断,在add tag后, 可以使用 if "aaa" in [tags]来判断tags里是否有aaa

\begin{lstlisting}
filter {
  if [foo] in [foobar] {
    mutate { add_tag => "field in field" }
  }
  if [foo] in "foo" {
    mutate { add_tag => "field in string" }
  }
  if "hello" in [greeting] {
    mutate { add_tag => "string in field" }
  }
  if [foo] in ["hello", "world", "foo"] {
    mutate { add_tag => "field in list" }
  }
  if [missing] in [alsomissing] {
    mutate { add_tag => "shouldnotexist" }
  }
  if !("foo" in ["hello", "world"]) {
    mutate { add_tag => "shouldexist" }
  }
}
\end{lstlisting}

\subsection{输出 output}
尽管 Elasticsearch 是我们的首选输出方向,能够为我们的搜索和分析带来无限可能,但它并非唯一选择,比如exec,file,hadoop等,详见\href{https://www.elastic.co/guide/en/logstash/current/output-plugins.html}{output plugins}


在output,也支持sprintf format,类似像统计一个状态数量便可以用到这个。平进经常用到的类似时间格式了,path => "/var/log/\%{type}.\%\{+yyyy.MM.dd.HH\}" 详细请参见\href{https://www.elastic.co/guide/en/logstash/current/event-dependent-configuration.html}{在配置中使用事件数据和字段}
\begin{lstlisting}
output {
  statsd {
    increment => "apache.%{[response][status]}"
  }
}
\end{lstlisting}

使用exec做为output,运行自定义脚本,需要先安装插件 \textit{ bin/logstash-plugin install logstash-output-exec}

\begin{lstlisting}
	output {
  if [type] == "abuse" {
    exec {
      command => "iptables -A INPUT -s %{clientip} -j DROP"
    }
  }
}
\end{lstlisting}

如果要在logstash使用geoip 可以安装相应插件bin/logstash-plugin install logstash-filter-geoip  并下载最新静态IP地址库到本地。https://dev.maxmind.com/geoip/geoip2/geolite2/  下载GeoLite2 City maxmindDB 放到指定目录

\section{kibana}




\end{document}
