\chapter{自动化管理工具}
服务器环境中要想保证其稳定运行,必不可少的便是标准化,自动化,设想任何一个运维人员都是上去手动修改主机配信息,一旦出故障,如果此运维人员还在职,且还记得修改过什么配置,还可以恢复回来,但恢复时长也相当长,这对IT管理造成相当大的困难,公司服务器标准化,自动化势在必行。
\section{ansible}
Ansible 基于python研发的自动化运维工具, ansible是无客户端也不需要启服务端工具,十分方便,主要基于openssl所以安全性也比较高,但是因为任务按队列依次执行,所以并没有saltsatck那样并发的快.特别是维护上百台机子后会感觉到明显慢很多。

ansible core : ansible 自身核心模块
host inventory: 主机库,定义可管控的主机列表
connection plugins: 连接插件,一般默认基于 ssh 协议连接
modules:core modules ( 自带模块 ) 、 custom modules ( 自定义模块 )
playbooks :剧本,按照所设定编排的顺序执行完成安排任务

收集配置文件

ansible -i environments/fudao/hosts  account -m fetch -a 'dest=/data/yulei/ansible-auto/roles/fudao/files/ src=/data/soft/nginx_80/conf/vhost/account.conf'

在play book里不仅可以指定vars变量,还可以指定vars 文件

var files

- hosts: myhosts
  vars_files:
    - default_step.yml


\section{salt}

\subsection{salt介绍}
与ansible不同的是,salt是一个C/S架构的软件,salt管理端为master,客户端叫minion,通过server端下发指令,客户端受指令的方式进行操作,saltstack基于zeromq消息队列来管理成千上万台主机客户端,传输指令执行相关操作。采用RSA key方式进行身份确认,传输采用AES方式进行加密,这使得它的安全性得到了保证。

在每个minion启动后便会自动生成RSA公密钥,存入于/etc/salt/pki/minion,中,根据minion配置文件中master地址,主动发送公钥给master等待master接收,master接收后便可以批量管理主机。下面是salt依赖的模块。

\begin{itemize}
\item msgpack-python, saltstack 消息交换库
\item YAML saltstack配置解析定义语法
\item Jinja2 saltstack 配置模板
\item MarkupSafe python unicode 转换库
\item apache-libcloud saltstack对去架构编排库
\item requests HTTP python 库
\item ZeroMQ saltstack消息系统
\item pyzqm zeromq python库
\item pycrypto python 密码库
\item m2crypto openssl python包装库
\end{itemize}

\subsection{salt中grains与pillar}

\subsubsection{Grains}

Grains 是saltstack组件之一,记录saltstack Minion 的一些静态信息的组件,(CPU, 内存, 磁盘, 网络, 等) 可以通过grains.items查看某台minion的所有Grain信息,minion的grains信息会在minions启动时汇报给master,在实际应用环境中我们需要根据自己的业务需求去算定义grains


查看grains函数 salt -E 'client*' sys.list_functions grains

自定义grians有三种方法: 
第一种在/etc/salt/master里直接配置,

\begin{lstlisting}
grains:
  roles:
    - webserver
    - memcache
\end{lstlisting}

第二种在另起一个文件/etc/salt/grains在里面定义,

\begin{lstlisting}
roles:
  - webserver
  - memcache
\end{lstlisting}

第三种使用python定义在minion配置文件中配置grains 放到任何环境中_grains目录下

\begin{lstlisting}
[root@linux-node1 /srv/salt/_grains]# cat my_grains.py
#!/usr/bin/env python
#-*- coding: utf-8 -*-

def my_grains():
    # 初始化一个grains字典
    grains = {}
    grains['iaas'] = 'openstack'
    grains['edu'] = 'sandow'
    return grains
[root@linux-node1 /srv/salt/_grains]# cat roles.py
#!/usr/bin/env python
#-*- coding: utf-8 -*-
import os.path
def roles():
    roles_file= "/etc/salt/roles"
    roles_list= []
    if os.path.isfile(roles_file):
        roles_fd = open(roles_file, "r")
        for eachroles in roles_fd:
            roles_list.append(eachroles[:-1])
    return {'roles': roles_list}
if __name__ == "__main__":
    print roles()
\end{lstlisting}

三种方法优先级为从高到低依次为系统自带,grains文件配置,master grains. 在每次修改完grains后需要同步更新grains.  salt '*' saltutil.sync_grains

\subsubsection{Pillar}
数据管理中心 Pillar
Pillar 也是salt组件之一,叫数据管理中心,会经常配合states在大规模配置管理工作中使用它,pillar在saltstack中主要的作用就是存储和定义配置管理中需要的一些数据,比如软件版本号,用户名,密码等信息,它的定义存储格式跟grains类似,
pillar 用来定位主机。 修改完pillar时需要刷新 salt '*' saltutil.refresh_pillar


给minion指定他想要的数据,查看pillar salt '*' pillar.items 
首先要把master里552行 改成True . pillar_opts: True 不过一般是不用于开启的。 
master 配置文件中指定了pillar的文件位置放置地方,
#pillar_roots:
# base:
# - /srv/pillar
所以我们可以在这个文件下面新建几个文件
[root@master-node1 pillar]# cat apache.sls

apache: httpd

apache: apache2


[root@master_101 pillar]# cat packages.sls
zabbix:
package-name: zabbix
version: 2.2.4
[root@master_101 pillar]# cat services.sls
zabbix:
port: 10050
user: admin
[root@master_101 pillar]# cat top.sls
base: #指定环境
'*': # target
- packages # 引用package.sls 或者packages/init.sls
- services # 引用services.sls
- apache
# 然后便可以使用命令来查看pillar啦
[root@master_101 pillar]# salt '*' pillar.item zabbix



grains 与 pillar的区别
名称	存储位置	数据类型	数据采集更新方式	应用
Grains	Minion端	静态数据	Minion启动时收集,也可以使用saltutil.sync_grans进行刷新	存储Mnion基本数据,比如用于匹配Minion,自身数据可以用来做资产管理等。
Pillar	Master端	动态数据	在master端定义,指定给对应的minion,可以使用saltutil.refresh_pillar刷新	存储master指定的数据,只有指定的Minion可以看到。用于敏感数据保存

===



\subsection{状态模块 state}

状态模块描述minion端的状态,按照官网的说明,往往最强大,最有用的工程解决方案都是基于简单的原则,
(Many of the most powerful and useful engineering solutions are founded on simple principles. Salt States strive to do just that: K.I.S.S. (Keep It Stupidly Simple))

salt state的核心便是sls文件(salt state file) sys文件描述了那些系统应该是什么样子。 sys是以yaml为格式序列化存储数据,所以其本质上就是字典,列表,数字,举个例子

\begin{lstlisting}
apache:
  pkg.installed: []
  service.running:
    - enable: True
    - require:
      - pkg: apache
\end{lstlisting}

这个sls状态文件将会确保apache已经安装,并且已经在运行。第一行apache是这个数据集的ID,全局惟一,一个ID下可以有多个模块,但是不能使用多次使用同一个模块。第二三行表示那些状态模块需要运行。基本模式是\textbf{<state_module>.<function>}, pkg.installed 确定当前主机已经安装了指定软件,如果不指定pkgs则默认安装第一行ID名。第三行service.running表示确保软件已经在运行。如果不指定name,默认以ID为软件名。
最后两行require表示service.running需要依赖于ID为apache下的pkg模块运行完后才会执行。 所以上面可以修改为

\begin{lstlisting}
testpkg:
  pkg.installed:
    - pkgs:
      - httpd
  service.running:
    - name: httpd
    - enable: True
    - require:
      - pkg: testpkg
\end{lstlisting}


要想运行状态文件需要在/etc/salt/master中开始file_roots配置
\begin{lstlisting}
file_roots:
  base:
    - /srv/salt
  dev:
    - /srv/salt/dev/services
    - /srv/salt/dev/states
  prod:
    - /srv/salt/prod/services
    - /srv/salt/prod/states
\end{lstlisting}

base, dev, prod表示环境,salt默认会去base环境下去找状态文件,假设把上面内容保存到/srv/salt/apache/init.sls, 要运行单个sls文件可以使用命令\textbf{salt '*' state.sls apache}运行,这里的apache,salt会去base环境下找apache.sls,如果没有,会继续找有没有目录apache,并且下面有init.sls,如果都没有则返回错误。

当apache目录下有多个sls时,可以使用include apache.xxx 来引到当前文件中

但是如果环境比较多,不同的主机运行不同的状态文件,你又不想一次次敲命令,又乱又容易弄错怎么办,这时候就出现top.sls,在整个salt状态文件里惟一,他定义了针对不同环境下不同主机运行不同的状态文件。默认放到base环境根目录下,也就是/srv/salt下。
\begin{lstlisting}
base:
  '*':
    - webserver
dev:
  'dev-*':
    - vim
  'db*dev*':
    - db
prod:
  '10.10.200.0/24':
    - match: ipcidr
    - deployments.qa.site1
\end{lstlisting}

最后使用命令\textbf{salt '*' state.highstate},一次搞定。

这里仅列出简单两个模块两个方法的用法,salt有非常多的模块可以使用,可以通过命令来获取所有模块以及模块中的方法及用途

\begin{description}
\item{查看state模块} 'Minion' sys.list_state_modules
\item{查看指定states(git)的所有functions} salt 'Minion' sys.list_state_functions git 
\item{查看指定function的用法}salt 'Minion' sys.state_doc git.config
\end{description}

执行结果处理
return组件可以理解为saltstack系统对执行minion返回后的数据进行存储或者返回给其他程序,
salt 状态模块
状态可以多次执行,


/etc/http/conf/http.conf; # 文件名
file.managed: # 模块.方法
- source: salt://apache/http.conf
- user: root
- group: root
- mode: 644
- template: jinja
- defaults:
custom_var: "default value"
other_var: 123

- context:
custom_var: 'override'



针对管理对象操作
module 是我们日常使用saltstack最多的一个组件,是用于管理对象操作的,这也是saltstack通过push的方式管理的入口,比如管理日常简单的执行命令,查看包安装情况,查看服务运行情况等都是通过 module来实现的 


这里为什么要加require?就是因为salt本身是并发的去处理任务,service和pkg有可能现时运行,这样有可能达不到预期结果,所以加着require做前后依赖。除require外,还有很多条件

1 我依赖谁 require
2 我被谁依赖 require_in
3 我监控谁 watch 监控某一个状态,如果状态变化 就做一些事
4 被谁监控 watch_in
5 我引用谁 include
6 我扩展谁  extend

\subsection{salt实践}

环境准备,使用两台主机做测试,并且在每一台机子都做hosts解析 

\begin{itemize}
\item master 端 主机名master_101 ip 172.16.1.101
\item minion 端 主机名client_102 ip 172.16.1.102
\end{itemize}

\begin{lstlisting}
#安装软件
yum install salt-master salt-minion -y

#master端
systemctl  start salt-master
systemctl  enable salt-master 

# minion端
sed -i 's/#master: salt/master: 172.16.1.101/g' /etc/salt/minion
systemctl  start salt-minion 
systemctl  enable salt-minion 
# 接收 公钥
salt-key -a client_102 -y
 Accepted Keys:
 client_102
 Denied Keys:
 Unaccepted Keys:
 Rejected Keys:
#测试连通性
salt 'client_102' test.ping
client_102:
True
\end{lstlisting}

salt 远程执行
目标(target)
模块(module)
返回(returnners)
salt '*' test.ping # salt target module.method key(test模块里的ping方法)
salt '*' cmd.run 'uptime'
目标
目标可以通过正则来匹配minion id. 或者用grains, pillar subnet/ip address, compound matching, node groups 来匹配


管理对象 target
saltstack系统中我们的管理对象叫作target,在master上我们可以采用不同的target 去管理不同的minion
在target options下可以分很多种匹配方式




===============================




架构布署
1。 系统初始化,
2 功能模块
3 业务模块
jinjia棋版
python的模版引擎,
jinja 模版来写keepalived的优写级

- ROUTEID: HAPROXY_MASTER
- STATEID: MASTER
- PRIORITYID: 101

- ROUTEID: HAPROXY_BACKUP
- STATEID: BACKUP
- PRIORITYID: 100



salt实践
分发文件
1. salt-cp 批量分发文件
salt-cp 语法格式为
salt-cp '*' [ options ] SOURCE DEST
salt-cp -E '.*' [ options ] SOURCE DEST
salt-cp -G 'os:Arch.*' [ options ] SOURCE DEST
salt-cp -G 'roles:nginx' conf/nginx /etc/init.d/
这样一条命令便可以把nginx文件分发到所有 roles中有nginx的服务器上
2. 通过状态模块分发文件
在正式环境下不使用 * 在执行状态的时候在后面加上 test=True
[root@test02 salt]# pwd
/srv/salt
[root@test02 salt]# ll conf/
total 20
-rwxr-xr-x 1 root root 1712 May 11 17:45 nginx
-rw-r--r-- 1 root root 2037 May 11 17:26 nginx.conf
-rw-r--r-- 1 root root 10361 May 10 10:56 zabbix_agentd.conf
[root@test02 salt]# cat nginx_conf.sls
/usr/local/nginx/conf/nginx.conf:
file.managed:
- name: /usr/local/nginx/conf/nginx.conf
- source: salt://conf/nginx.conf
- user: root
- group: root
- mode: 644
[root@test02 salt]# salt -G 'roles:nginx' state.sls nginx_conf
如果分发文件里面的配置有一些是动态的,那么就需要使用 jinja 模版
[root@test02 salt]# cat zabbix_agent.sls
/etc/zabbix/zabbix_agentd.conf:
file.managed:
- name: /etc/zabbix/zabbix_agentd.conf
- source: salt://conf/zabbix_agentd.conf
- user: root
- group: root
- mode: 644
- template: jinja
- defaults:
hostname: grains['fqdn']
[root@test02 salt]# grep '^[a-Z]' conf/zabbix_agentd.conf
PidFile=/var/run/zabbix/zabbix_agentd.pid
LogFile=/var/log/zabbix/zabbix_agentd.log
LogFileSize=0
Server=10.51.48.249
ServerActive=10.51.48.249:10051
Hostname= {{ grains['fqdn'] }}
Include=/etc/zabbix/zabbix_agentd.d/

[root@test02 salt]# salt -G 'roles:nginx' state.sls zabbix_agent

文件管理之为文件增加一些东西
salt ‘*’ file.append /etc/hosts ‘this is add line’
salt
zeromq 消息队列
状态不能回滚
apache-install: #id 声明(在高级状态下id必须唯一), name 声明
1 安装软件包 pkg
2 修改配置文件 file
3 启动服务 service
网络 系统 数据库 去计算 自动化 架构 安全
一个id声明下面,状态模块不能重复使用
common_packges:
pkg.installed:
- pkgs:
- unzip
- dos2unix
- salt-minion: 2015.8.5-1.el6
lamp-pkg:
pkg.installed:
- pkgs:
- httpd
- php
- marriadh-server
- php-mysql
- php-cli
- php-mbstring
/etc/http/conf/http.conf:
file.managed:
- source: salt://apache/http.conf
- user: root
- group: root
- mode: 644
- template:jinja
- defaults:
php-config:
file.managed:
- name: /etc/php.ini
- source: salt://files/php.ini
- user: root
- group: root
- mode: 644
php-config:
file.managed:
- name: /etc/my.cnf
- source: salt://files/my.cnf
- user: root
- group: root
- mode: 644

如何编写sls技巧:
1 按状态分类 如果单独使用,清晰
2 按服务分类,可以被其他的sls include.
jinja2
{}
{{[ … ]}} 变量
3 步走:
1 告诉file模块, 你要使用jinja
template: jinja
2 你要列出参数列表
* defaultes:
PORT: 88

模板里面支持 salt grains pillar 进行赋值
salt 运程执行模块 {{ salt[‘network.hw_addr’](‘eth0’) }}
{{ pillar[‘apache’]}}

1 系统初始化
2 功能模块: 设置单独的目录 haproxy nginx php mysql memcached
3 业务模块: 根据业务类型划分,例如web 服务,论坛,bbs
干活: 1 salt环境配置
开发,测试,(功能测试环境,性能测试环境) 预生产,生产
base 基础环境
init目录: 环境初始化 1 dns配置, 2 history记录时间, 3 记录命令操作 4 内核参数优化 5 安装yum仓库 6 安装zabbix-agent
socket 网络套接字

file.manage
backup:minion
会把minion文件备份到下面目录下
/var/cache/salt/minion/file_baup/
继续学习状态间关系
1. unless
2. onlyif
把变量写到最上面,然后在下面引用

name: {{ keepalived_tar }}
session
对于php程序来说,只要在../php/lib/php.ini文件里面配置即可session.save_handler = files【session存储形式—files或者memcache】;session.save_path = “tcp://IP:Port”【session的存储路径】,他的session默认是保存在/tmp目录下,但是我们将他们都挂载到后端的memcache内存缓存服务器中。
例如:
session.save_handler = memcache
session.save_path = “tcp://127.0.0.1:11211” 或者”tcp://host1:port1?persistent=1&weight=2,tcp://host2:port2”
对于java程序来说,一般的session共享也是有一个memcache服务,比较多的情况下使用程序来实现session共享,通过一个变量将session记下来,然后存到memcache里面,需要的时候在取。
另外一般比较大的java程序一般都会使用cookie,将session存到客户端,这样情况下就不需要共享,只要我本地设置了cookie就可以了
将nginx加到systemd服务中
cat > /usr/lib/systemd/system/nginx.service << ‘EOF’
[Unit]
Description=nginx - high performance web server
Documentation=http://nginx.org/en/docs/
After=network.target remote-fs.target nss-lookup.target
[Service]
Type=forking
PIDFile=/run/nginx.pid
ExecStartPre=/usr/sbin/nginx -t -c /etc/nginx/nginx.conf
ExecStart=/usr/sbin/nginx -c /etc/nginx/nginx.conf
ExecReload=/bin/kill -s HUP 

MAINPID 
PrivateTmp=true
[Install]
WantedBy=multi-user.target
EOF
master_job_cache: mysql
mysql.host: ‘192.168.56.11’
mysql.user: ‘salt’
mysql.pass: ‘salt@pw’
mysql.db: ‘salt’
mysql.port: 3306
systemctl restart salt-master
https://www.unixhot.com/docs/saltstack/ref/modules/all/salt.modules.saltutil.html#module-salt.modules.saltutil
1 修改minion配置文件
file-client ; local
file_roots:
piller_roots
2 编写SLS
3 salt-call –local sate.highstate
https://www.unixhot.com/docs/saltstack/ref/netapi/all/salt.netapi.rest_cherrypy.html
SALT-API
1 https 证书
2 配置证书
3
useradd -M -s /sbin/nologin saltapi
passwd saltapi
[root@linux-node1 /etc/pki/tls/certs]# make testcert
cd /etc/pki/tls/private
openssl rsa -in localhost.key -out salt_nopass.key
systemctl start salt-api
pip install CherryPy==3.2.6
[root@linux-node1 ~]# vim /etc/salt/master
default_include: master.d/*.conf
rest_cherrypy:
host: 192.168.56.11
port: 8000
ssl_crt: /etc/pki/tls/certs/localhost.crt
ssl_key: /etc/pki/tls/private/salt_nopass.key
external_auth:
pam:
saltapi:
- .*
- ‘@wheel’
- ‘@runner’
[root@linux-node1 ~]# curl -k https://192.168.56.11:8000/login \ 
-H ‘Accept: application/x-yaml’ \
-d username=’saltapi’ \
-d password=’saltapi’ \
-d eauth=’pam’
return:
* eauth: pam
expire: 1464663850.123221
perms:

    * .*
    * ‘@wheel’
    * ‘@runner’
start: 1464620650.123221
token: 785db9bc5e79dee828bfb1649bc49c59900e0ebf
user: saltapi
curl -k https://192.168.56.11:8000/minions/linux-node1.oldboyedu.com \ 
-H ‘Accept: application/x-yaml’ \
-H ‘X-Auth-Token: 785db9bc5e79dee828bfb1649bc49c59900e0ebf’ \
curl -k https://192.168.56.11:8000/ \ 
-H ‘Accept: application/x-yaml’ \
-H ‘X-Auth-Token: 785db9bc5e79dee828bfb1649bc49c59900e0ebf’ \
-d client=’runner’ \
-d fun=’manage.status’
curl -k https://192.168.56.11:8000/ \ 
-H ‘Accept: application/x-yaml’ \
-H ‘X-Auth-Token: 785db9bc5e79dee828bfb1649bc49c59900e0ebf’ \
-d client=’local’ \
-d tgt=’*’ \
-d fun=’test.ping’
https://github.com/binbin91/oms

